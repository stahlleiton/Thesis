\section{Dataset and simulated samples} \label{sec:WBoson_Samples}

\subsection{Dataset} \label{sec:WBoson_Samples_Data}

The primary dataset used for the \W analysis is PASingleMuon. The dataset contains events from \pPb collisions at 8.16~\TeV recorded by the CMS detector requiring at least one identified muon. The data were reconstructed with the CMS software version 8.0.30 and was thoroughly validated by the CMS collaboration. Only runs passing the data quality requirements were processed. The total integrated luminosity of the recorded data corresponds to 173.4~\nbinv, currently known within 5$\%$~\cite{LUMI}.

In the first part of the \pPb run (labelled as \RunPbp), the proton was heading towards negative pseudorapidity $-\eta$, according to the CMS detector convention~\cite{CMS}, with an energy of 6.5~\TeV and colliding a lead nuclei at 2.56~\TeV~\cite{LHC}. In the second part of the \pPb run (labelled as \RunpPb), the proton was going towards $+\eta$, still boosted compared to the lead nuclei. The integrated luminosity of the \RunPbp and \RunpPb runs were 62.6~\nbinv and 110.8~\nbinv, respectively.

%%------------------------------------------------------------%%
\subsection{Next-to-leading order simulations} \label{sec:WBoson_Sample_MC}

Fully reconstructed Monte Carlo (MC) simulated samples are used to describe the signal and the different sources of Electro-Weak (EWK) backgrounds. The MC samples were generated at Next-to-Leading order (NLO) using the POsitive Weight Hardest Emission Generator (\POWHEG)~\cite{POWHEG,POWHEG_2,POWHEGBOX} version 2. To account for Quantum Chromo-Dynamic (QCD) and EWK corrections, the \POWHEGBOX  packages \verb#W_ew-BMMNP#~\cite{POWHEGBOX_W_ew_BMNNP} and \verb#Z_ew-BMMNPV#~\cite{POWHEGBOX_Z_ew_BMNNP} were used to generate the $\pp\to\W\to l\PGnl$ and $\pp\to\DY\to l^{+}l^{-}$ processes, respectevely. The $\pp\to\ttbar$ was generated using the \POWHEGBOX package \verb#hvq#~\cite{POWHEGBOX_hvq}, which is a heavy flavour quark generator at NLO QCD. 

Since the data is based on \pPb collisions, the simulation of both proton-proton (\pp) and proton-neutron (\pn) collisions is crucial. In order to simulate \pPb collisions and also include nuclear modifications to the PDFs, the \POWHEG framework was modified. This was done by first generating \pp collisions using the CT14 parton distribution function~\cite{CT14}. Then each PDF was corrected by applying the EPPS16 nuclear modification factors derived for Pb ions~\cite{EPPS16}. And finally, the \cPqu-quark and \cPqd-quark PDFs were scaled considering the number of neutrons and protons in the Pb nucleus. The nuclear modification and normalization of the quark PDFs is shown in \eq{eq:PbScaling}.

\begin{equation}
\begin{aligned}
f^{\cPqu}_{Pb} &= \frac{Z}{A}\left[R^{\cPqu}_{s}f^{\cPaqu}_{\Pp} + R^{\cPqu}_{v}\left(f^{\cPqu}_{\Pp}-f^{\cPaqu}_{\Pp}\right)\right] + \frac{A-Z}{A}\left[R^{\cPqd}_{s}f^{\cPaqd}_{\Pp} + R^{\cPqd}_{v}\left(f^{\cPqd}_{\Pp}-f^{\cPaqd}_{\Pp}\right)\right] \\
f^{\cPqd}_{Pb} &= \frac{Z}{A}\left[R^{\cPqd}_{s}f^{\cPaqd}_{\Pp} + R^{\cPqd}_{v}\left(f^{\cPqd}_{\Pp}-f^{\cPaqd}_{\Pp}\right)\right] + \frac{A-Z}{A}\left[R^{\cPqu}_{s}f^{\cPaqu}_{\Pp} + R^{\cPqu}_{v}\left(f^{\cPqu}_{\Pp}-f^{\cPaqu}_{\Pp}\right)\right]
\end{aligned}
\label{eq:PbScaling}
\end{equation}

where $f$ represent the quark PDF, $R_{s}$ ($R_{v}$) is the nuclear modification factor for sea (valence) quarks, A is the Pb mass number and Z is the Pb atomic number.

The parton showering is performed by hadronizing the \POWHEG events in \PYTHIA 8.212~\cite{PYTHIA8} with the CUETP8M1 underlying event (UE) tune~\cite{PYTHIA_TUNE,UE_pp}. The decay of \PGt particles is handled in \PYTHIA using the \TAUOLA C++ Interface 1.1.5~\cite{TAUOLA}, including final state radiative (FSR) QED corrections using~\PHOTOS 2.15 \cite{PHOTOS}. The full CMS detector response is simulated in all MC samples, based on \GEANTfour~\cite{GEANT4}, considering a realistic aligment and calibration of the beam spot and the different sub-detectors of CMS, tuned on data. The MC events are reconstructed with the standard CMS \pp reconstruction software used during 2016 data taking.

To consider a more realistic distribution of the underlying environment present in the \pPb collisions, the MC signal events were embedded in a minimum bias sample generated with \EPOS~\cite{EPOS}, taking into account the \pPb boost direction. The EPOS MC samples were tuned to reproduce the global event properties of the \pPb data such as the charged-hadron transverse momentum spectrum and the particle multiplicity \cite{dNdEta_pPb}. The list of simulated samples and the corresponding cross sections used are summarized in \tab{tab:MCSamples}.

\begin{table} [h!]
  \centering
  \resizebox{\textwidth}{!}{
    \renewcommand{\arraystretch}{1.5}
    \begin{tabular}{c c c c c c}
      \hline
      Process & Criteria & Generator & PDF & Cross section (nb) & Events \\
      \hline
      $\pPb \to \WToMuNuPl$ & & \POWHEG+\PYTHIA & CT14+EPPS16 & 1213.4 (NLO) & 982714  \\
      $\Pbp \to \WToMuNuPl$ & & \POWHEG+\PYTHIA & CT14+EPPS16 & 1214.1 (NLO) & 981874  \\
      $\pPb \to \WToMuNuMi$ & & \POWHEG+\PYTHIA & CT14+EPPS16 & 1082.2 (NLO) & 995726  \\
      $\Pbp \to \WToMuNuMi$ & & \POWHEG+\PYTHIA & CT14+EPPS16 & 1083.4 (NLO) & 998908  \\
      \hline
      $\pPb \to \WToTauNuPl$ & & \POWHEG+\TAUOLA & CT14+EPPS16 & 1146.3 (NLO) & 481125  \\
      $\Pbp \to \WToTauNuPl$ & & \POWHEG+\TAUOLA & CT14+EPPS16 & 1147.4 (NLO) & 500000  \\
      $\pPb \to \WToTauNuMi$ & & \POWHEG+\TAUOLA & CT14+EPPS16 & 1026.3 (NLO) & 495450  \\
      $\Pbp \to \WToTauNuMi$ & & \POWHEG+\TAUOLA & CT14+EPPS16 & 1019.4 (NLO) & 498092  \\
      \hline
      $\pPb \to \DYToMuMu$ & $10<M_{\DY}<30$ & \POWHEG+\PYTHIA & CT14+EPPS16 & 1182.2 (NLO) & 997120  \\
      $\Pbp \to \DYToMuMu$ & $10<M_{\DY}<30$ & \POWHEG+\PYTHIA & CT14+EPPS16 & 1168.0 (NLO) & 1000000 \\
      $\pPb \to \DYToMuMu$ & $30<M_{\DY}<\infty$ & \POWHEG+\PYTHIA & CT14+EPPS16 & 266.3 (NLO) & 1000000  \\
      $\Pbp \to \DYToMuMu$ & $30<M_{\DY}<\infty$ & \POWHEG+\PYTHIA & CT14+EPPS16 & 266.3 (NLO) & 1000000  \\
      \hline
      $\Pbp \to \DYToTauTau$ & $10<M_{\DY}<30$ & \POWHEG+\TAUOLA & CT14+EPPS16 & 1143.7 (NLO) & 464494  \\
      $\Pbp \to \DYToTauTau$ & $30<M_{\DY}<\infty$ & \POWHEG+\TAUOLA & CT14+EPPS16 & 259.4 (NLO) & 498444  \\
      \hline
      $\pPb \to \ttbar$ & & \POWHEG+\PYTHIA & CT14+EPPS16 & 45 (CMS) & 99578  \\
      $\Pbp \to \ttbar$ & & \POWHEG+\PYTHIA & CT14+EPPS16 & 45 (CMS) & 100000  \\
      \hline
    \end{tabular}
  }
  \caption{Simulated NLO samples used for the \W boson measurement in \pPb at 8.16~\TeV. The listed cross sections are the \POWHEG generator cross sections scaled by 208 (atomic number of Pb ion), except for \ttbar cross sections which are taken from the latest measurement in \pPb at 8.16\TeV by CMS \cite{HIN-17-002}.}
  \label{tab:MCSamples}
\end{table}
  
%%------------------------------------------------------------%%
\subsection{Center of Mass Frame} \label{sec:WBoson_Sample_CMFrame}
  
Due to the energy difference between the \pPb colliding beams, the nucleon-nucleon center-of-mass (CM) frame was not at rest with respect to the laboratory (LAB) frame. Massless particles emitted at a pseudorapidity $\eta_{CM}$ in the CM frame experienced a longitudinal boost as described in equation \ref{eq:CMShift}.

\begin{equation}
\abs{\Delta{\eta}_{CM}} = \frac{1}{2}\times\left|\ln\left(\frac{Z_{Pb}\times A_{p}}{Z_{p}\times A_{Pb}}\right)\right| = \frac{1}{2}\times\ln\left(\frac{208}{82}\right) = 0.465
\label{eq:CMShift}
\end{equation}

The shift between the CM and LAB frames depends on the orientation of the beams, and are applied in the following way:

\begin{itemize}
\item \RunPbp\ (1st run): $\eta_{LAB} = \eta_{CM} - 0.465$
\item \RunpPb\ (2nd run): $\eta_{LAB} = \eta_{CM} + 0.465$
\end{itemize}


%%------------------------------------------------------------%%
\subsection{Combining \pPb runs} \label{sec:WBoson_Sample_CombiningBeamDirection}

Since the CMS detector is symmetric with respect to the beam orientation, the \RunpPb and \RunPbp samples are merged in order to maximize the statistics of the MC simulations and the data. This is done by first flipping the sign of the pseudorapidity of particles from the \RunPbp sample measured in the laboratory frame, and then combine them with the events from the \RunpPb sample. In the case of the MC samples, the events are reweighed before merging them, so that the simulated luminosity match the integrated luminosity recorded in each proton-lead run. The combined sample is labelled as "pA" and corresponds to \pPb collisions with the proton always going toward $+\eta$.


% END OF SUBSECTION




