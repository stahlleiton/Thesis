\subsection{Object Reconstruction} \label{sec:WBoson_Analysis_ObjectReconstruction}

This section provides details on how the main objects are identified and selected. The \pPb data at $\sqrtsnn = 8.16$~\TeV was reconstructed with the standard algorithms implemented in \CMSSW release \verb#8_0_24# and used during the LHC Run 2 \pp runs. Among the reconstructed objects, muons and the Missing Transverse Energy (MET), are used in this analysis to extract the signal and suppress the background.


%%------------------------------------------------------------%%
\subsubsection{Missing Transverse Energy} \label{sec:WBoson_Analysis_MET}

Since neutrinos are not detected in the CMS detector, their presence is characterized by a particle momentum imbalance in the transverse plane, also called Missing Transverse Energy (\ETslash). 

The MET is defined as the negative vectorial sum of the transverse momemtum of all reconstructed particles. Its coordinates along the the x and y axes in the CMS coordinate system \cite{CMS}, can be computed as:

\begin{eqnarray*} 
\ETslash(x) &=& - \sum_{particles} \pt \times \cos(\phi) \\
\ETslash(y) &=& - \sum_{particles} \pt \times \sin(\phi) \\
\ETslash &=& \sqrt{\ETslash(x)^2 + \ETslash(y)^2}
\end{eqnarray*}

If a neutrino is produced, its $\vec{p}_{T}$ would be opposite to the total momentum of all the other particles in the event due to momemtum conservation. Therefore, the momentum of the neutrino would be exactly the same as the \ETslash.

The particle-flow algorithm (PF) \cite{PF_Reco} is used to identify the particles in the event and estimate the $\ETslash$. The PF algorithm is designed to reconstruct all stable particles such as electrons, muons, photons, charged hadrons and neutral hadrons, by taking into account all CMS sub-detectors. The outcome is an optimal determination of each particle's type, momentum and energy. This set of particles is then used to measure the $\vec{\Em}_{T}$. The performance of the MET reconstruction in \pp data has been documented in \cite{MET_Reco,MET_PERF}.

Ideally, in an event where no neutrinos are produced, the \ETslash should be equal to zero. But since the momentum of particles is not measured with perfect precision, the sum of the reconstructed particles $\vec{p}_{T}$ does not cancel completely due to the resolution of the detector. In order to correct for the differences in the MET resolution between MC and data, the resolution of the hadronic recoil component is smeared in MC to match the data as explained in \sect{sec:WBoson_Analysis_RecoilCorrection}.


%%------------------------------------------------------------%%
\subsubsection{Online Muon Selection} \label{sec:WBoson_Analysis_MuonTrigger}

The muon triggers are divided in two systems: the Level-1 (L1) trigger and the High Level Trigger (HLT). 

The L1 triggers are hardware-based and were updated to Stage-2 during 2016 (LHC Run 2) using $\mu$TCA technology. The Stage-2 L1 muon trigger system is divided in 3 sub-systems: Endcap, Overlap and Barrel Muon Track Finders. The muon candidates found by each track finder are sent to the L1 Global Muon Trigger ($\mu$GMT), which process them and selects the best muon candidates. Afterwards, the L1 Global Trigger ($\mu$GT) takes the decision to either accept or reject the event based on the information provided by the $\mu$GMT and the calorimeters. In order to only consider real collisions, all \pPb L1 trigger algorithms were required to be in coincidence with a bunch crossing identified by the Beam Pick-up Timing for eXperiments (BPTX) detector. The technical information of the upgraded L1 muon trigger system is described in detail in \cite{L1_Muon_Stage2_Paper,L1_Muon_Stage2_Thesis}. 

The HLT are software-based and are only applied once an event is accepted by the L1 trigger system. The HLT muon triggers implemented for \pPb data are divided in 3 levels: the \verb#HLT_PAL1# trigger takes as input all events fired by the L1 muon trigger and does not apply any further cuts, the \verb#HLT_PAL2# trigger reconstructs the muon candidates found in the muon sub-detectors using a more sophisticated algorithm compared to L1, and the \verb#HLT_PAL3# trigger uses the muon tracks reconstructed by combining the inner tracker hits and muon tracks found by \verb#HLT_PAL2#. The HLT muon reconstruction algorithms were identical to the ones used during the 2016 \pp runs. A complete description of the CMS HLT trigger system can be found in \cite{CMS_Trigger}.

For this analysis, events are selected using the muon trigger path \verb#HLT_PAL3Mu12_v*#. This trigger requires a fully reconstructed online muon with $\pt > 12$~\GeVc. The HLT was seeded by the L1 trigger path \verb#L1_SingleMu7#, which pass events with at least one L1 muon with $\pt > 7$~\GeVc. The muon trigger was unprescaled both at L1 and HLT levels during the entire data taking period.

If in a given event, the main analysis trigger \verb#HLT_PAL3Mu12_v1# fired and a reconstructed muon is matched to the HLT L3 muon that fired the trigger, the reconstructed muon is considered trigger matched.

The matching criteria between the reconstructed and the HLT muon is done by requiring that $\Delta{R}(\mu_{reco} , \mu_{HLT}) < 0.1$.


%%------------------------------------------------------------%%
\subsubsection{Muon Identification} \label{sec:WBoson_Analysis_MuonIdentification}

During the \pPb data taking, muon tracks were reconstructed using the \pp muon reconstruction algorithm \cite{Muon_Reco}. High-\pt muons are identified in this analysis through the use of the \textit{Tight ID} cuts endorsed by the Muon POG for Run 2 analyses. The full description of the tight selection is included in \url{https://twiki.cern.ch/twiki/bin/viewauth/CMS/SWGuideMuonIdRun2#Tight_Muon}.

The tight muon selection includes the following variable cuts, which are applied on the reconstructed muons:

\begin{itemize}
\item isGlobalMuon : Selects prompt high-\pt muons and rejects decays in flight, punch-throughs and accidental matching.
\item isPFMuon : Selects muons reconstructed with the Particle-Flow algorithm \cite{PF_Reco}.
\item Global track $\chi^{2}$/NDF $< 10$ : Requires a reasonable global muon fit quality.
\item Number of valid muon hits $> 0$ : Requires at least one valid hit in the muon detectors associated with the global muon, making sure that the information from the tracker and the muon system is consistent.
\item Number of matched muon stations $> 1$ : Requires segments in at least two muon stations making the selection consistent with the muon trigger logic.
\item $\abs{d_{xy}} < 0.2$~cm : Requires a transverse impact parameter consistent with the primary vertex to reduce cosmic background and muons from decays in flight.
\item $\abs{d_{Z}} < 0.5$~cm : Requires a longitudinal distance consistent with the primary vertex to further suppress cosmic muons, muons from decays in flight and tracks from Pile-Up.
\item Number of valid pixel hits $> 0$ : Requires at least one valid pixel hit to further suppress muons from decays in flight.
\item Number of tracker layers with measurement $> 5$ : Requires at least 6 tracker layers with hits to guarantee a good \pt measurement.
\end{itemize}

Any difference in the performance of the muon cuts observed between simulation and data, is corrected in MC through the use of the Tag-And-Probe scale factors as described in \sect{sec:WBoson_Analysis_CorrectedEfficiency}.


%%------------------------------------------------------------%%
\subsubsection{Muon Isolation} \label{sec:WBoson_Analysis_MuonIsolation}

Muon isolation allows to select muons from a \W or \Z boson decay, around which little activity is expected, and to reject muons inside jets, which are surrounded by a large number of particles. There are various ways to define this isolation, depending on what type of particles or energy we include in the sum, what is the size of the cone we consider, etc.

In order to optimise the definition of the isolation in \pPb collisions, a study has been made in data. The selection is the following:

  \begin{itemize}
   \item Tight muons with $\pt>15$~\GeVc, $|\eta|<2.4$
   \item Signal: opposite sign dimuon with $80<M_{\mu\mu}<100$~\GeVcc
   \item Background sample 1 ("SS"): same sign events with $80<M_{\mu\mu}<100$~\GeVcc
   \item Background sample 2 ("low MET"): single muon events with \ETslash $<5$~ \GeVc (to reduce the W boson "contamination").
  \end{itemize}
  
Both background samples are dominated by multi-jet production. The "SS" sample has closer kinematics to the signal but has a smaller number of events compared to the "low MET" sample.

In any case, a relative isolation should be used (of the form  $\text{iso} = (\sum_\text{in cone} \pt) / \pt^\mu$). The reason is that higher \pt muons in jets will have more energy around them.
We have tested many possible variables for the isolation: summing only certain PF IDs, cone size 0.3 or 0.4, etc.) It turns out that the best performance\footnote{defined as the area below the ROC curve, (1-bkg. eff.) vs (sig. eff.)} is obtained with the default PF isolation, without pile-up correction (see also \fig{fig:isoPFR03NoPUCorr}):

 \begin{equation}
 \label{eq:iso}
  \text{isolation} = \left(\sum_\text{charged hadrons, neutral hadrons, photons} \pt\right) / \pt^\mu
 \end{equation}

\begin{figure}[htbp]
 \begin{center}
  \includegraphics[width=0.44\textwidth]{Figures/WBoson/Analysis/ObjectReconstruction/Muon/Isolation/PF_Muon_IsoPFR03NoPUCorr.pdf}\hspace{0.1\textwidth}
  \includegraphics[width=0.44\textwidth]{Figures/WBoson/Analysis/ObjectReconstruction/Muon/Isolation/PF_Muon_IsoPFR03NoPUCorr_rocSS.pdf}
 \end{center}
 
 \caption{\label{fig:isoPFR03NoPUCorr} Left: efficiency as a function of the PF isolation cut defined in \eq{eq:iso}, with a cone of 0.3 and without isolation, for opposite sign data (black), Drell-Yan MC (red), same-sign data (blue) and low MET single muon data (green). Right: ROC curve, using the opposite-sign data as signal and same-sign data as background.}
\end{figure}


Now that we have determined that the best performance is obtained with the default PF isolation, we need to also check what the optimal cone size is. The recommended cone size for isolation for muons in
Run~2 pp collisions is 0.4\footnote{\url{https://twiki.cern.ch/twiki/bin/view/CMS/SWGuideMuonIdRun2##Muon_Isolation}}, but one could possibly expect the optimal cone size to be smaller for \pPb collisions, where there is much more activity from the underlying event, affecting indifferently the W boson signal and the multi-jet background. Therefore, we now vary the cone size and study the impact on the performance. Different cone sizes are tested, from 0.1 to 0.45, as shown in \fig{fig:isocones}.

The optimal depends on which is the considered background sample, but a cone size of 0.3 seems to be a good working point. This is also the cone size that was chosen for the Run~1 \pPb W boson analysis \cite{HIN-13-007}.
 
 \begin{figure}[htbp]
 \begin{center}
  \includegraphics[width=0.5\textwidth]{Figures/WBoson/Analysis/ObjectReconstruction/Muon/Isolation/conesizes.pdf}
 \end{center}
 
 \caption{\label{fig:isocones} Area under the ROC cruve (indicative of the performance of the isolation) as a function of the cone size, for two choices of background: same-signa data (blue diamonds) or low MET single muon data
 (magenta squares).}
\end{figure}

\begin{figure}[htbp]
 
 \begin{center}
 \includegraphics[width=0.31\textwidth]{Figures/WBoson/Analysis/ObjectReconstruction/Muon/Isolation/PF_Muon_myIsoPFR020.pdf}
 \includegraphics[width=0.31\textwidth]{Figures/WBoson/Analysis/ObjectReconstruction/Muon/Isolation/PF_Muon_myIsoPFR030.pdf}
 \includegraphics[width=0.31\textwidth]{Figures/WBoson/Analysis/ObjectReconstruction/Muon/Isolation/PF_Muon_myIsoPFR040.pdf}
 
 \includegraphics[width=0.31\textwidth]{Figures/WBoson/Analysis/ObjectReconstruction/Muon/Isolation/PF_Muon_myIsoPFR020_rocSS.pdf}
 \includegraphics[width=0.31\textwidth]{Figures/WBoson/Analysis/ObjectReconstruction/Muon/Isolation/PF_Muon_myIsoPFR030_rocSS.pdf}
 \includegraphics[width=0.31\textwidth]{Figures/WBoson/Analysis/ObjectReconstruction/Muon/Isolation/PF_Muon_myIsoPFR040_rocSS.pdf}
 \end{center}
 
 \caption{\label{fig:isoConesSS}The efficiency as a function of the isolation cut (top) and ROC curve with a same-sign data background (bottom) for a cone size of 0.2 (left), 0.3 (middle) or 0.4 (right).}
\end{figure}

\begin{figure}[htbp]
 
 \begin{center}
 \includegraphics[width=0.31\textwidth]{Figures/WBoson/Analysis/ObjectReconstruction/Muon/Isolation/PF_Muon_myIsoPFR020.pdf}
 \includegraphics[width=0.31\textwidth]{Figures/WBoson/Analysis/ObjectReconstruction/Muon/Isolation/PF_Muon_myIsoPFR030.pdf}
 \includegraphics[width=0.31\textwidth]{Figures/WBoson/Analysis/ObjectReconstruction/Muon/Isolation/PF_Muon_myIsoPFR040.pdf}
 
 \includegraphics[width=0.31\textwidth]{Figures/WBoson/Analysis/ObjectReconstruction/Muon/Isolation/PF_Muon_myIsoPFR020_roc.pdf}
 \includegraphics[width=0.31\textwidth]{Figures/WBoson/Analysis/ObjectReconstruction/Muon/Isolation/PF_Muon_myIsoPFR030_roc.pdf}
 \includegraphics[width=0.31\textwidth]{Figures/WBoson/Analysis/ObjectReconstruction/Muon/Isolation/PF_Muon_myIsoPFR040_roc.pdf}
 \end{center}
 
 \caption{\label{fig:isoConesLowMET}The efficiency as a function of the isolation cut (top) and ROC curve with a low MET single muon data background (bottom) for a cone size of 0.2 (left), 0.3 (middle) or 0.4 (right).}
\end{figure}

It was also studied whether the pile-up had an effect on the isolation, and whether this possible effect should be mitigated with the use of a pile-up correction. This correction is used in pp analyses and consists in removing from the isolation variable the expected contribution from particles coming from other primary vertices. Though the charged PF contribution only comes from the primary vertex of interest, the neutral hadron PF and the photons cannot distinguish between the different vertices and include contributions from all primary vertices in the event. One can look at the charged PF coming from other primary vertices to have an idea of this contribution. Thus, the pile-up correction ($\beta=0.5$) to the isolation takes into account this effect. However, the average pile-up in 2016 \pPb was around $\langle\mu\rangle = 1$, while it was about $\langle\mu\rangle = 23$ in 2016 pp data, so effects due to pile-up does not have a significant impact on the muon isolation in \pPb data.

The distribution and ROC curve for the isolation variable is shown in \fig{fig:isonoPU} for events with a single primary vertex and \fig{fig:isoPU} for events with at least 2 primary vertices.
Again we see that the performance does not depend much on whether there are several primary vertices or not, and the pile-up correction does not help (it even seems like the opposite).

\begin{figure}[!htbp]
  
 \includegraphics[width=0.49\textwidth]{Figures/WBoson/Analysis/ObjectReconstruction/Muon/Isolation/PU1/PF_Muon_IsoPFR03_roc.pdf}
 \includegraphics[width=0.49\textwidth]{Figures/WBoson/Analysis/ObjectReconstruction/Muon/Isolation/PU1/PF_Muon_IsoPFR03NoPUCorr_roc.pdf}
 
 \caption{\label{fig:isonoPU} The ROC curve for events with exactly one reconstructed primary vertex, with (left) and without (right) pile-up correction.}

\end{figure}

\begin{figure}[!htbp]
 
 \includegraphics[width=0.49\textwidth]{Figures/WBoson/Analysis/ObjectReconstruction/Muon/Isolation/PUGt1/PF_Muon_IsoPFR03_roc.pdf}
 \includegraphics[width=0.49\textwidth]{Figures/WBoson/Analysis/ObjectReconstruction/Muon/Isolation/PUGt1/PF_Muon_IsoPFR03NoPUCorr_roc.pdf}

 \caption{\label{fig:isoPU} The ROC curve for events with at least two reconstructed primary vertices, with (left) and without (right) pile-up correction.}
\end{figure}

\textbf{To conclude}, it was found that optimal isolation variable to be used is the standard PF isolation, with a cone size of 0.3, and without pile-up correction. The optimal cut on this variable 
(defined as the cut simultaneously maximising the signal efficiency and background rejection) was found to be 0.15. The final isolation variable is:

\begin{equation}
 \text{isolation} = \left(\sum_{\text{charged hadrons in} DR<0.3} \pt + \sum_{\text{neutral hadrons in} DR<0.3} \pt + \sum_{\text{photons in} DR<0.3} \pt\right) / \pt^\mu
\end{equation}



%%------------------------------------------------------------%%
\subsubsection{Gen-Reco Matching} \label{sec:WBoson_Analysis_GenRecoMatching}

A generated muon is considered matched to a reconstructed muon if the following criteria are fulfilled:

\begin{itemize}
\item Both muons must have the same charge
\item Eta-Phi cone size of $\Delta{R}(\mu_{gen} , \mu_{reco}) < 0.5$
\item $\delta{\pt}(\mu_{gen} , \mu_{reco}) = \frac{\Delta{\pt}(\mu_{gen} , \mu_{reco})}{\pt(\mu_{gen})} < 0.5$
\end{itemize}


%%------------------------------------------------------------%%
\subsubsection{Leading Muon} \label{sec:WBoson_Analysis_LeadingMuon}

Since events can have more than one muon, a criteria is defined to choose the best muon candidate in each event. In this analysis, the leading muon is used to bin the data and is determined as explained below.

First, a set of reconstructed muons passing the Tight identification cuts (see \sect{sec:WBoson_Analysis_MuonIdentification}) within $|\eta| < 2.4$ are selected. In Monte-Carlo, the list of selected muons are also required to be matched to a generated muon using the criteria mentioned in \sect{sec:WBoson_Analysis_GenRecoMatching}. The matching is done to exclude reconstructed muons coming from the underlying EPOS embedded events, since their generated information is not stored in the MC samples. Afterwards, the muon with the highest \pt among all the selected muons is chosen as the leading muon of the event.

The event is only kept if the leading muon has a $\pt > 25$~\GeVc and it is matched to an online muon that fired the HLT trigger as described in \sect{sec:WBoson_Analysis_MuonTrigger}.



% END OF SUBSECTION
\clearpage
