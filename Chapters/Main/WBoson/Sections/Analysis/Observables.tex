\subsection{Observables} \label{sec:WBoson_Analysis_Observables}

The main motivation behind measuring the \Wb-boson production in \RunpPb collisions is to probe the nuclear modifications of the PDFs. To accomplish this, the efficiency-corrected \WToMuNu event yields are combined to measure three kinds of observables: cross sections, muon charge asymmetry and forward-backward ratios.

\paragraph{\WToMuNu cross sections.} The \WToMuNupm differential cross sections are computed as a function of \etaMuCM, according to:

\begin{equation}
 \frac{\dd\sigma{(\WToMuNupm)}}{\dd\etaMuCM}\left(\etaMuCM\right) = \frac{N^{\pm}_{\PGm}\left(\etaMuCM\right)}{\Lumi \cdot \Delta{\etaMuCM}}
 \label{eq:CrossSection}
\end{equation}

where $\Lumi = 173.4{\pm}6.1$~\nbinv is the recorded integrated luminosity, $\Delta\etaMuCM$ is the width of the \etaMuCM range in which the measurement is performed and $N_{\PGm}\left(\etaMuCM\right)$ is the number of signal events after correcting for efficiency.

\paragraph{Muon charge asymmetry.} The muon charge asymmetry measures the difference between the event yields of the \WToMuNuMi and \WToMuNuPl processes, which is sensitive to the number of protons and neutrons in the nucleus (isospin effect), and to the flavour dependence of the nuclear modifications of the PDFs. It is defined in the following way:

\begin{equation}
 \ChgAsym\left(\etaMuCM\right) = \frac{N_{\PGm}^{+}\left(\etaMuCM\right) - N_{\PGm}^{-}\left(\etaMuCM\right)}{N_{\PGm}^{+}\left(\etaMuCM\right) + N_{\PGm}^{-}\left(\etaMuCM\right)}
 \label{eq:MuonChargeAsymmetry}
\end{equation}

where $N_{\PGm}^{-}$ and $N_{\PGm}^{+}$ represents the efficiency-corrected number of \WToMuNuMi and \WToMuNuPl events, respectively.

\paragraph{Forward-backward ratios.} To probe the modification of the PDFs between different pseudorapidity regions, the signal event yields  measured in the forward region ($\etaMuCM > 0$) are combined with those measured in the backward region ($\etaMuCM < 0$), to derive forward-backward ratios. These ratios are computed separately for \WToMuNuPl and \WToMuNuMi events in the following way:

\begin{equation}
 \RFBpm\left(\etaMuCM\right) = \frac{ N^{\pm}_{\PGm}\left(+\etaMuCM\right) }{ N^{\pm}_{\PGm}\left(-\etaMuCM\right) }
 \label{eq:MuonForwardBackwardAsymmetryCharged}
\end{equation}

A forward-backward ratio is also derived for all \WToMuNu events, by combining the yields of the \WToMuNuMi and \WToMuNuPl processes, according to:

\begin{equation}
 \RFB\left(\etaMuCM\right) = \frac{ N^{+}_{\PGm}\left(+\etaMuCM\right) + N^{-}_{\PGm}\left(+\etaMuCM\right) }{ N^{+}_{\PGm}\left(-\etaMuCM\right) + N^{-}_{\PGm}\left(-\etaMuCM\right) }
 \label{eq:MuonForwardBackwardAsymmetry}
\end{equation}

% END OF SUBSECTION