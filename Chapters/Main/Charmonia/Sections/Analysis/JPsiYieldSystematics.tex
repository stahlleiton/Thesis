\subsection{Systematic uncertainties of \texorpdfstring{\JPsi}{J/psi}-meson yields}\label{sec:Charmonia_Analysis_JPsiYieldSystematics}

This section describes the procedure used to derive the systematic uncertainties associated to the measurement of the prompt and nonprompt \JPsi-meson yields. The different sources of systematic uncertainties arise from: the parametrisation of the dimuon invariant mass and pseudoproper-decay length  distributions used to extract the signal, and the estimation of the efficiency of \JPsi mesons. In this case, the leading systematic uncertainty of the \JPsi-meson measurements in \Runpp and \RunPbPb collisions correspond to the \tnp efficiency corrections and the \ctau parametrisation, respectively. When describing each systematic source, the corresponding largest relative uncertainty is mentioned.

\subsubsection{Uncertainty on the dimuon invariant mass parametrisation}\label{sec:Charmonia_Analysis_JPsiYieldSystematics_InvMass}

The uncertainty associated to the modelling of the \mMuMu distribution arise from the parametrisation of the signal and background invariant mass shape. It is determined by varying the different components of the \mMuMu functional form and redoing the 2D data fits, while using the nominal \ctau functional form.

\paragraph{Parametrisation of the \texorpdfstring{\JPsi}{J/psi}-meson invariant mass distribution.} In order to estimate the systematic uncertainty associated to the choice of the \JPsi-meson invariant mass shape, two variations are performed:
\begin{itemize}
 \item \textbf{Variation of the \JPsi-meson invariant mass parameters}: the parameters fixed to simulation (i.e. the tail parameters $\aJPsi$ and $\nnJPsi$, and the ratio of CB widths) are released and the data fits are repeated. To improve the convergence of the fits, a set of Gaussian penalty functions, centred in the nominal value of the corresponding parameters, are added to constrain their range of variation.

The range of variation of $\aJPsi$, $\nnJPsi$ and the ratio of CB widths is determined from data by redoing the \mMuMu fits, leaving only one parameter free at the time while the other parameters are fixed to their nominal values. The RMS of the difference between the parameter value extracted from the data fit and the nominal one is computed including the results from different \ptMuMu and centrality intervals within each rapidity region, and the largest RMS obtained among the different rapidity regions is taken as the width of the corresponding Gaussian penalty function. The RMS is defined here as:

\begin{equation}
 \text{RMS} = \sqrt{\frac{1}{\sum_{i}{1/(\sigma_{\text{data}}^{i})^2}} \cdot \sum_{i}{\frac{(\text{par}_{\MC} - \text{par}_{\text{data}}^{i})^2}{(\sigma_{\text{data}}^{i})^2}}}
\end{equation}

where the sum runs over different \pt and centrality bins in the same rapidity region, $\text{par}_{\text{data}}^{i}$ and $\sigma_{\text{data}}^{i}$ is the value and uncertainty of the parameter extracted from the data fit in a given analysis bin $i$, respectively, and $\text{par}_{\MC}$ is the corresponding nominal value derived from simulation. The width of the Gaussian penalty functions of each CB parameter is presented, relative to the nominal parameter value, in \tab{tab:parsRMS}.

\begin{table}[htb!]
  \centering
  \begin{tabular}{lcccc}
    \hline\hline
    System & $\aJPsi$ [\%] & $\nnJPsi$ [\%] & ${\sigma_{\CB,2}/\sigma_{\CB,1}}$ [\%] \\
    \hline
    \Runpp   & 16 & 21 &     \\
    \RunPbPb & 21 & 54 & 30  \\
  \end{tabular}
  \caption{Relative width used in the Gaussian penalty functions for the tail parameters $\aJPsi$ and $n$, and the ratio of CB widths. The Gaussian width is shown relative to the nominal parameter value.}
  \label{tab:parsRMS}
\end{table}

The systematic uncertainty associated to the determination of the signal mass parameters from simulations is then estimated by performing the data fits with the Gaussian penalty functions, and the difference between the varied \JPsi-meson yields and the nominal results is taken as the uncertainty.

 \item \textbf{Variation of the \JPsi-meson invariant mass model}: the functional form of the signal invariant mass shape is changed from the nominal double Crystal Ball function to a Crystal Ball plus a Gaussian function (with common mean parameters), defined as:

\begin{equation}
  M_{\JPsi}\left(m_{\mumu}\right) = \fJPsi \cdot \CB\left(\mMuMu\right) + \left(1- \fJPsi\right) \cdot \Gauss\left(\mMuMu\right)
\end{equation}

As in the nominal case, the parameters of the alternative signal \mMuMu model are tuned from fits to the \JPsi-meson simulations, and the tail parameters \aJPsi and \nnJPsi are fixed to simulation in both \Runpp and \RunPbPb data fits, while the ratio of CB over Gaussian widths ($\sigma_{\CB}/\sigma_{\text{G}}$) is fixed only in the \RunPbPb data fits.

\end{itemize}

The systematic uncertainty on the signal shape parametrisation is then determined from the quadratic sum of the uncertainties obtained from varying the invariant mass parameters and the shape model. The corresponding uncertainty for prompt and nonprompt \JPsi mesons amounts to 1.2\% (2.7\%) and 2.4\% (2.7\%), in \Runpp (\RunPbPb) collisions, respectively.

\paragraph{Parametrisation of the background invariant mass shape.} In the case of the background invariant mass parametrisation, three variations are performed to derive the corresponding systematic uncertainty, given by:

\begin{itemize}
 \item \textbf{Variation of the LLR test threshold}: the $\chi^{2}$ probability threshold is increased from the nominal value (5\%) to 10\% and reduced to 2.5\%, and the LLR tests are repeated for each threshold. The background models selected from each LLR test are then used to redo the 2D fits.
 
 \item \textbf{Variation of the dimuon invariant mass fitting range}: the range of the \mMuMu distribution used in the 2D fits is changed from 2.6-3.5~\GeVcc to 2.6-3.4~\GeVcc. The 2D fits are then remade using the same orders of Chebyshev functions as used in the nominal fits.
 
 \item \textbf{Variation of the background invariant mass model}: the background \mMuMu functional form is changed from the nominal Chebyshev function to an exponential of a Chebyshev function, defined as:

\begin{equation}
 M_{\bkg}^{N}\left(\mMuMu\right) = \exp\left[\sum_{i=0}^{N} {c_{i} T_{i}\left(\mMuMu\right)}\right]
\end{equation}

where $T$ are Chebyshev polynomials and $c$ are free parameters. As in the nominal analysis, the \mMuMu distribution in data is fitted using the alternative background model with orders between 0 and 6, and the LLR test is employed with a 5\% threshold to decide the best order in each analysis bin.
  
\end{itemize}

The uncertainty associated to each systematic variation is determined by computing the deviation of the measured prompt and nonprompt \JPsi-meson yields from the nominal results. In the case of the two variations done for the LLR test threshold, the maximum deviation between the two variations is taken for each \JPsi-meson yield. The systematic uncertainties of the different sources are combined by adding them in quadrature. The combined uncertainty amounts to 0.6\% (3.0\%) for prompt \JPsi mesons and 1.6\% (2.9\%) for nonprompt \JPsi mesons, in \Runpp (\RunPbPb) collisions.


\subsubsection{Uncertainty on the pseudoproper-decay length parametrisation}\label{sec:Charmonia_Analysis_JPsiYieldSystematics_Ctau}

The different systematic variations performed for the pseudoproper-decay length parametrisation are summarised as follows:

\begin{enumerate}
 \item Modelling of the \sigmactau distribution: replace the nominal signal and background \sigmactau templates in the 2D fits with the template of the total \sigmactau distribution.
 \item Parametrisation of the \ctau resolution: parametrise the \ctau resolution model using the simulated sample of prompt \JPsi mesons instead of data.
 \item Parametrisation of the nonprompt \JPsi-meson \ctau shape: replace the exponential \ctau model used to describe nonprompt \JPsi mesons in the 2D fits with a template of the \ctau distribution derived from simulation.
 \item Parametrisation of the background \ctau shape: use a template of the \ctau distribution from the background-like dataset derived with \sPlot, instead of a functional form.
\end{enumerate}

The method and result of these four sources are detailed below, and the resulting uncertainties are summed in quadrature with the other systematic sources.

\paragraph{Modelling of the \sigmactau distribution.} To estimate the uncertainty associated to the use of the signal and background \sigmactau template histograms, derived from the \sPlot background- and signal-like datasets, the template histograms of the signal and background are made instead using the full \sigmactau distribution and the 2D fits are remade. The difference between the varied and nominal \JPsi meson yields is taken as the systematic uncertainty, which amounts to 0.6\% (6.3\%) for prompt \JPsi mesons and 2.1\% (4.2\%) for nonprompt \JPsi mesons, in \Runpp (\RunPbPb) collisions.

\paragraph{Parametrisation of the \ctau resolution.} The systematic uncertainty due to the parametrisation of the \ctau resolution is estimated by extracting the \ctau resolution parameters from \JPsi-meson simulations instead of the data. The \ctau resolution parameters are extracted from simulated samples of prompt \JPsi mesons by fitting the nominal \ctau resolution model (weighed sum of three Gaussians) to the simulated \ctau distribution. The varied \ctau resolution parameters are then used to remake the 2D fits and the uncertainty is derived from the difference between the varied \JPsi-meson yields and the nominal ones. This systematic uncertainty amounts in \Runpp (\RunPbPb) collisions to 1.5\% (4.7\%) for prompt \JPsi mesons and 5.3\% (9.6\%) for nonprompt \JPsi mesons.

\paragraph{Parametrisation of the nonprompt \JPsi-meson \ctau shape.} The systematic uncertainty associated to the modelling of the \ctau distribution of nonprompt \JPsi mesons is computed by replacing the nominal signal functional form (convolution of exponential decay with \ctau resolution model) with an unbinned template of the reconstructed \ctau distribution derived from simulation of nonprompt \JPsi mesons. The \ctau templates are made using a kernel estimation technique~\cite{RooKeysPDF}, implemented in the RooFit framework, which parametrise the distribution of a given variable by superimposing a Gaussian function to each data point. The uncertainty is then determined from the difference between the varied signal yields and the nominal results, and reaches up to 0.8\% (3.4\%) for prompt \JPsi mesons and 2.1\% (12.1\%) for nonprompt \JPsi mesons, in \Runpp (\RunPbPb) collisions.

\paragraph{Parametrisation of the background \ctau shape.} The systematic uncertainty related to the choice of background \ctau model is determined by replacing the nominal background functional form (three exponential decay functions convolved with the \ctau resolution model) with an unbinned template. The template is built from the \ctau distribution of the \sPlot background-like dataset employed in \sect{sec:Charmonia_Analysis_JPsiYieldExtraction_CtauPar}, using the RooFit kernel estimation technique. This uncertainty contributes in \Runpp (\RunPbPb) collisions up to 0.5\% (10\%) for prompt \JPsi mesons and 1.2\% (22.3\%) for nonprompt \JPsi mesons.


\subsubsection{Uncertainty on the \texorpdfstring{\JPsi}{J/psi}-meson efficiency}\label{sec:Charmonia_Analysis_JPsiYieldSystematics_Efficiency}

There are two main sources of systematics that affects the measurement of the \JPsi-meson efficiencies: the \tnp-correction weights used to correct the simulated efficiencies and the charmonium $\ptMuMu$-$\rapMuMu$ weights applied to improve the modelling of the \JPsi-meson \pt and rapidity. Among these two, the largest uncertainty is obtained from the \tnp corrections, which is dominated by the uncertainty on the extraction of the standalone-muon reconstruction efficiency in data.

\paragraph{Tag-and-probe correction.} The uncertainty associated to the \tnp correction derives from the measurement of the \tnp data efficiency of muon identification, trigger, tracking and standalone-muon reconstruction.

Regarding the tracking efficiency, an overall systematic uncertainty is determined from the largest difference found between data and simulation, which corresponds to a relative uncertainty on the \JPsi-meson yields measured in \Runpp and \RunPbPb collisions of 1.0\% and 2.4\%, respectively. On the other hand, for the standalone-muon reconstruction, trigger and muon identification, the uncertainties of the \tnp-correction weights are separated in a statistical and systematic component. The statistical component of the \tnp-correction uncertainty is evaluated by producing a hundred sets of \tnp-correction weights, where each point is randomly generated using a Gaussian distribution spread according to its statistical uncertainty. The hundred sets of \tnp-correction weights are then used to recompute the \JPsi-meson efficiencies and the corresponding systematic uncertainty is estimated by computing the RMS of the hundred variations of the prompt and nonprompt \JPsi-meson efficiencies. The systematic component of the \tnp-correction uncertainty is propagated using two sets of \tnp-correction weights generated by shifting all points up and down, according to the systematic uncertainty of each point (derived by varying the settings of the \tnp invariant mass fits). The \JPsi-meson efficiencies are then corrected with each set of \tnp-correction  weights and the maximum deviation of the two varied efficiencies with respect to the nominal one is taken as the systematic uncertainty on the efficiency of prompt and nonprompt \JPsi mesons.

The statistical component of the \tnp uncertainty represents 1.5\% (4.6\%) for the \Runpp (\RunPbPb) efficiencies. Regarding the systematic components of the \tnp uncertainty, the largest uncertainty is obtained from the standalone-muon reconstruction efficiency, which corresponds to 9.6\%, while the \tnp-correction uncertainties associated to the trigger and muon identification efficiencies in \Runpp (\RunPbPb) collisions amounts to 0.5\% (5.2\%) and to 1.1\% (3.3\%), respectively.

\paragraph{Charmonium transverse momentum and rapidity weighing.} The simulated samples of \JPsi mesons are weighed as a function of dimuon \pt and rapidity, to match the \pt spectrum of prompt and nonprompt \JPsi mesons observed in data in four rapidity regions. In order to estimate the uncertainty of the weighing procedure, a hundred sets of weights are randomly generated using a Gaussian distribution for each $\ptMuMu$-$\rapMuMu$ interval, where the Gaussian width is fixed to the uncertainty of the corresponding dimuon kinematic weight. The simulations are reweighed with each set of generated dimuon kinematic weights and then used to recompute the efficiencies of prompt and nonprompt \JPsi mesons. The corresponding systematic uncertainty is then determined from the RMS of the hundred varied efficiencies compared to the nominal efficiency for prompt and nonprompt \JPsi mesons. In this case, the largest relative uncertainty on the \Runpp (\RunPbPb) \JPsi-meson efficiencies corresponds to 0.2\% (1.8\%).


% END OF SUBSECTION