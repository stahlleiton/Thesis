\let\textcircled=\pgftextcircled
\chapter*{Introduction} \label{sec:Introduction}
\addcontentsline{toc}{chapter}{Introduction}

The progress made by the scientific community over the last century has pushed the boundaries of our understanding of the subatomic world and led to the formulation of one the most successful theories of physics, corresponding to the Standard Model (SM) of particle physics. Even though the SM framework is able to describe, with great accuracy, the interactions and properties of most known particles, some fundamental phenomena still need to be clarified, such as the phase states of matter or the evolution of particles in a nuclear environment.

Under normal circumstances, the main constituents of matter, called partons (i.e. quarks and gluons), are confined by the strong nuclear force into hadrons. However, at high enough temperatures or densities, matter undergoes a phase transition to a state where quarks and gluons become asymptotically free, known as the Quark Gluon Plasma (QGP). Such extreme state of matter is believed to have prevailed during the first microseconds of the creation of the universe and to be part of the core of neutron stars. To recreate the QGP in the laboratory, heavy ions are collided in accelerator facilities at high energies. The QGP can be probed in heavy-ion experiments by measuring different observables, such as the production yield of particles that interact strongly with the QGP medium (e.g. quarkonia, jets, ...). In addition, the environment present in a nucleus can also affect the production of particles produced in heavy-ion collisions, even in the absence of QGP. The measurement of electroweak particles that do not interact with the QGP medium (photons, \Z and \Wb bosons) allows to study the nuclear modification of Parton Distribution Functions (PDF). The PDFs of nuclei are crucial inputs to theory predictions for heavy-ion colliders and their precise determination with experimental data is indispensable for calculations of the initial stage of nucleus-nucleus reactions.

Three analyses are presented in this thesis. All of them use data recorded by the Compact Muon Solenoid~\cite{CMS} apparatus at the Large Hadron Collider~\cite{LHC}. The first one measure the production of \Wb bosons in \RunpPb collisions at a center-of-mass energy per nucleon pair of $\sqrtsnn = \SI{8.16}{\TeV}$, with the goal to provide precise experimental constrains to the nuclear modifications of the quark PDFs. I am the \textit{contact person} of this analysis and have conducted all the work except the tag-and-probe and the weak boson \pt corrections. I presented the preliminary results at the Quark Matter~\cite{QM2018} and ICHEP~\cite{ICHEP2018} conferences in 2018. The work is expected to be published in a peer-reviewed journal in the near future~\cite{HIN-17-007}. The second and third analyses probe quark deconfinement in the QGP by measuring the \JPsi and \PsiP (i.e. charmonium) production in \RunPbPb collisions at $\sqrtsnn = \SI{5.02}{\TeV}$. My main contributions to the \JPsi and \PsiP analyses include the optimization of the muon kinematic selection, the signal extraction and the systematic uncertainties associated to the fitting. The results of the \PsiP and \JPsi analyses have been published in PRL~\cite{CMS_Psi2S_PbPb_5p02TeV} and EPJC~\cite{CMS_JPsi_PbPb_5p02TeV}, respectively, and I presented them at the Hot Quarks 2016~\cite{HQ2016} and EPS-HEP 2017~\cite{EPS2017} conferences.

The manuscript is organised as follows. The general concepts of the strong interactions and heavy-ion collisions are introduced in \chp{sec:Physics}. A brief description of the main probes of the QGP concludes the chapter. \chp{sec:Experiment} describes the experimental apparatus, where the operational conditions of the Large Hadron Collider and characteristics of the Compact Muon Solenoid detector are detailed. The chapter also describes the trigger and reconstruction algorithms employed to select and process the data. \chp{sec:WBoson} presents in details the generated samples, the event selections, the corrections to the missing transverse momentum, the estimation of the muon efficiency, the signal extraction, the systematic uncertainties and the results of the \Wb-boson analysis, accompanied by a short introduction on electroweak physics. The charmonium analysis in \RunPbPb collisions is exposed in \chp{sec:Charmonia}. The chapter contains details on the charmonium samples, the event selection, the \JPsi efficiency estimation, the extraction of the \JPsi yields and the \PsiP/\JPsi ratios, the systematic uncertainties and the results, including a brief introduction to the physics of charmonia in heavy-ion collisions.


% END OF CHAPTER
\clearpage

\chapter{High energy nuclear physics} \label{sec:Physics}

% INTRODUCTION OF CHAPTER

\initial{T}his chapter introduces some key concepts of high energy nuclear physics common to the analysis of the production of {\PW} bosons and charmonia in heavy-ion collisions. The quantum field theory of the strong interactions is described in \sect{sec:Physics_SI}. The state of hot dense hadronic matter, known as the quark-gluon plasma, and the study of its properties in heavy-ion collisions are reviewed in \sect{sec:Physics_HI}.

% SECTIONS OF CHAPTER
\section{Physics}\label{sec:Introduction_Physics}

% END OF SECTION


% END OF CHAPTER
\clearpage
