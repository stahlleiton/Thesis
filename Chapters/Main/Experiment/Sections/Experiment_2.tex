\section{Experimental setup}\label{sec:Introduction_Experiment}

\subsection{Large Hadron Collider}\label{sec:Introduction_Experiment_LHC}


The Large Hadron Collider (LHC) is a massive circular particle collider with a circunference of 26.7~km located 175~m underground beneath the borders between France and Switzerland. The construction of the LHC by the European Organization for Nuclear Research (CERN) took 10 years and was completed in 2008. The LHC is designed to accelerate two beams of particles in opposite directions along two different adjacent pipes. There are four interaction points where the particle beams collide corresponding to each of the four main LHC experiments: A Toroidal LHC ApparatuS (ATLAS), the Compact Muon Solenoid (CMS), A Large Ion Collider Experiment (ALICE) and the LHC beauty (LHCb). The trajectory of the particle beams are controlled using superconducting magnets operated at a temperature of 1.9~K using liquid Helium-4. Superconducting dipole magnets are used to bend the beams and keep the particules circulating while quadrupole magnets are used to focus the beams increasing the probability of collisions.

The LHC programme underwent an operational run (labelled as run 1) between 2009 and 2013, followed by a upgrade period of 2 years (LS1) and is currently at the end of the second operation run. Run 1  started with proton-proton ({\PpPp}) collisions at 2.36~\TeV in 2009, followed by {\PpPp} collisions at 7~\TeV and lead-lead collisions at 2.56~\TeV between 2010 and 2011, {\PpPp} collisions at 8~\TeV in 2012, proton-lead collisions at 5.02~\TeV in 2013. Run 2 started with {\PpPp} collisions at 13~\TeV and {\PbPb} collisions at 5.02~\TeV in 2015, \pPb collisions at 8.16~\TeV in 2016, {\PpPp} collisions at 5.02~\TeV in 2017, Xenon-Xexon collisions at 5.16~\TeV and will finish with {\PbPb} collisions at 5.02~\TeV at the end of 2018.


Protons are extrated from a hydrogen gas after stripping the electrons from the hydrogen atoms using an electric field. The protons are then accelerated in the Linear Accelerator facility 2 (Linac 2) to an energy of 50~\MeV. The proton beam is subsequently accelerated to 1.4~\GeV by the Proton Synchroton Booster (PSB), and then to 25~\GeV by the Proton Synchroton (PS). After been accelerated by the PS, the beam is transfered to the Super Proton Synchroton (SPS) where the protons are accelerated reaching an energy of 450~\GeV and are then sent to the LHC where they reach a maximum energy of 6.5~\TeV.

Ion beams are created using an Electron Cyclotron Resonance (ECR) ion source, and then accelerated through a Radio Frequency Quadrupole (RFQ) and the heavy-ion linear collider LINAC 3. The ions are stripped from their electrons by a carbon foil creating ions of $\mathrm{Pb}^{54}$. The acceleration and cooling of the Pb ions is then performed in the Low Energy Ion Ring (LEIR). The ion beams are then injected in the PS and then follow the same acceleration chain as the protons upwards.


\subsection{CMS detector}\label{sec:Introduction_Experiment_CMS}




\subsection{Trigger System}


\subsection{Object reconstruction}



% END OF SECTION
