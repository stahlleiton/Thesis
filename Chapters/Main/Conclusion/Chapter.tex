\chapter*{Conclusion} \label{sec:Conclusion}
\addcontentsline{toc}{chapter}{Conclusion}

The understanding of the cold nuclear matter effects, arising from the sole presence of nuclei, is crucial in order to characterise the quark-gluon plasma (QGP) produced in heavy-ion collisions. Among these effects, one that impacts the production of particles formed in the initial hard scattering is the nuclear modification of the parton distribution functions (PDF). Due to the non-perturbative behaviour of the strong interactions, the PDFs can not be determined theoretically and instead are parametrised using experimental data. Weak bosons provide good measurements of the PDF modifications in nuclear collisions since they do not interact strongly with the medium. Thanks to the high collision energy available at the Large Hadron Collider (LHC), it has become possible to measure the production of weak bosons in heavy-ion collisions. The LHC collaborations have studied the weak boson production in \RunpPb at $\sqrtsnn = \SI{5.02}{\TeV}$, where hints of nuclear modifications of the PDFs where observed in the forward rapidity region, although the free-proton PDF calculations were also consistent with the measurements within the statistical precision of the data.

In the scope of this thesis, I measured the \PW-boson production in \RunpPb collisions at $\sqrtsnn = \SI{8.16}{\TeV}$ with the Compact Muon Solenoid detector and required the systematic uncertainties to be largely decreased. Compared to previous measurements at $\sqrtsnn = \SI{5.02}{\TeV}$, the analysis benefits from an increased \PW-boson statistics due to the higher beam energy and integrated luminosity. The measured \PW-boson production is found to be in good agreement with the EPPS16 and nCTEQ15 nuclear PDF sets. On the other hand, the \PW-boson measurements significantly disfavoured the CT14 free-proton PDF calculations, revealing unambiguously the presence of nuclear modifications in the production of electroweak bosons, for the first time. Considering the smaller size of the measured uncertainties, compared to the model calculations, the \PW-boson measurements have the  potential to constrain the parametrisations of the quark nuclear PDFs.

The hot nuclear matter effects caused by the QGP are probed in this thesis through the study of the charmonium production in heavy-ion collisions. Two related analyses were presented in the second part of the manuscript: the production of prompt and nonprompt \JPsi mesons, and the nuclear modification of \PsiP mesons relative to \JPsi mesons, in \RunPbPb collisions at $\sqrtsnn = \SI{5.02}{\TeV}$.

Prompt and nonprompt \JPsi mesons are found to be suppressed in all measurements. Their nuclear modification factor (\raa) is observed to depend on centrality, being more suppressed towards more central collisions, while no significant dependence on rapidity is seen. On the one hand, an indication of weaker suppression is observed for prompt \JPsi mesons, in the lowest transverse momentum (\pt) interval ($3 < \pt < 6.5$~\GeVc) and most central collisions (0-30\%), which may originate from \JPsi regeneration. Also, for the first time, a hint of less suppression of prompt \JPsi mesons is seen in the highest \pt range ($\pt > 25$~\GeVc) compared to the intermediate \pt range ($6.5 < \pt < 25$~\GeVc), which may reflect the energy loss of initial partons fragmenting into \JPsi mesons. On the other hand, the nonprompt \JPsi-meson suppression is seen to be more pronounced at high \pt, likely caused by jet quenching of bottom quarks. In the overlapping range, the measured \JPsi-meson \raa is compatible with previous measurements at $\sqrtsnn = \SI{2.76}{\TeV}$.

The measurement of the \doubleRatio double ratio in \RunPbPb at $\sqrtsnn = \SI{5.02}{\TeV}$ shows that the \PsiP mesons are more suppressed than \JPsi mesons, which is consistent with the sequential suppression of charmonia in the QGP. Comparisons with measurements at $\sqrtsnn = \SI{2.76}{\TeV}$ show a good agreement of the double ratio at high \pt in the mid-rapidity region. On the contrary, extending the \pt range down to 3~\GeVc in the forward rapidity region shows a stronger reduction of the double ratio at $\sqrtsnn = \SI{5,02}{\TeV}$ compared to $\sqrtsnn = \SI{2.76}{\TeV}$, where the two measurements deviates by almost 3 standard deviations in the centrality-integrated interval. A sequential regeneration of charmonia has been suggested to explain the double ratio results at both collision energies.

% END OF CHAPTER