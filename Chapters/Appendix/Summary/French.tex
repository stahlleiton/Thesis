\chapter{R{\'e}sum{\'e} en fran{\c{c}}ais}\label{app:Summary_French}


\section*{Chapitre 1{~:} Physique nucl{\'e}aire des hautes {\'e}nergies}

Les progr{\`e}s r{\'e}alis{\'e}s par la communaut{\'e} scientifique au cours du si{\`e}cle dernier ont repouss{\'e} les limites de notre compr{\'e}hension du monde subatomique et ont conduit {\`a} la formulation d'une des th{\'e}ories de la physique les plus abouties, le mod{\`e}le standard (SM) des particules. Le SM est un cadre th{\'e}orique qui d{\'e}crit les propri{\'e}t{\'e}s des particules {\'e}l{\'e}mentaires et leurs interactions. Les particules {\'e}l{\'e}mentaires sont soit des fermions, soit des bosons.

Les fermions sont des particules de spin demi-entier qui se comportent selon les statistiques de Fermi-Dirac. Ils sont class{\'e}s en deux cat{\'e}gories{~:} les leptons et les quarks. Il existe six leptons organis{\'e}s en trois \textit{g{\'e}n{\'e}rations}{~:} l'{\'e}lectron (\PGem) et le neutrino {\'e}lectronique (\PGnGe), le muon (\PGmm) et le neutrino muonique (\PGnGm), et le tau (\PGtm) et le neutrino tauique (\PGnGt). Dans le cas des quarks, il existe six \textit{saveurs} coupl{\'e}es {\'e}galement en trois g{\'e}n{\'e}rations{~:} up (\cPqu) et down (\cPqd), charm (\cPqc) et strange (\cPqs), top (\cPqt) et bottom (\cPqb).

Les interactions entre les fermions sont d{\'e}crites dans le SM par trois interactions fondamentales{~:} la force {\'e}lectromagn{\'e}tique, la force nucl{\'e}aire forte et la interaction faible. Chaque interaction fondamentale est m{\'e}di{\'e}e par l'{\'e}change de bosons, qui sont des particules de spin entier qui suivent les statistiques de Bose-Einstein. Les interactions {\'e}lectromagn{\'e}tiques entre les particules avec une charge {\'e}lectrique sont m{\'e}di{\'e}es par des photons. Les interactions faibles peuvent agir sur tous les fermions et sont m{\'e}di{\'e}es par les bosons vecteurs {\Wp}, {\Wm}  et {\PZ}. Les interactions fortes sont m{\'e}di{\'e}es par les gluons et sont d{\'e}crites par la th{\'e}orie de la chromodynamique quantique (QCD).

La chromodynamique quantique est une th{\'e}orie de champ quantique non ab{\'e}lienne reposant sur le groupe de sym{\'e}trie de jauge SU(3). Les principaux objets de la QCD sont les quarks qui portent une charge de couleur et les gluons qui portent une charge de couleur et une charge d'anti-couleur. La force des interactions fortes est param{\'e}tr{\'e}e par la constante de couplage $\alpha_{s}$, qui d{\'e}pend de l'{\'e}chelle d'{\'e}nergie $Q$. Aux faibles {\'e}nergies, le couplage fort augmente, confinant les partons (i.e. les quarks et les gluons) dans des hadrons, tandis qu'{\`a} des {\'e}nergies plus {\'e}lev{\'e}es, le couplage fort diminue, ce qui conduit {\`a} la libert{\'e} asymptotique des partons (les partons apparaissent comme ponctuels et libres quand on interagit sur un hadron {\`a} grande {\'e}nergie). Les hadrons compos{\'e}s de trois (anti-)quarks sont appel{\'e}s (anti-)baryons, alors que ceux compos{\'e}s d'un quark et d'un anti-quark sont appel{\'e}s des m{\'e}sons.

La production de particules lors de collisions hadroniques d{\'e}pend de l'{\'e}volution des partons {\`a} l'int{\'e}rieur des hadrons. Le contenu partonique des hadrons peut {\^e}tre {\'e}tudi{\'e} via les fonctions de distribution de parton (PDF), qui repr{\'e}sentent la probabilit{\'e} qu'un parton porte une fraction donn{\'e}e $x$ ({\'e}galement appel{\'e}e Bjorken $x$) de la quantit{\'e} de mouvement totale du hadron. Selon le th{\'e}or{\`e}me de factorisation de la QCD, la section efficace d'un processus dure donn{\'e} peut {\^e}tre scind{\'e}e en une section efficace partonique multipli{\'e}e par la PDF de chaque hadron entrant. D'une part, la section efficace partonique peut {\^e}tre calcul{\'e}e {\`a} l'aide de la m{\'e}thode QCD perturbative et ne d{\'e}pend pas des hadrons en collision. D'autre part, les PDF ne peuvent pas {\^e}tre calcul{\'e}s {\`a} partir des premiers principes en raison de la nature non perturbative de la QCD, mais ils peuvent {\^e}tre d{\'e}termin{\'e}s {\`a} partir d'ajustements globaux reposant sur des donn{\'e}es exp{\'e}rimentales, car les PDF sont ind{\'e}pendants du processus de diffusion initial.

La mati{\`e}re nucl{\'e}aire normale existe dans la nature {\`a} basse temp{\'e}rature et {\`a} densit{\'e} baryonique relativement {\'e}lev{\'e}e. Cependant, {\`a} des temp{\'e}ratures ou des densit{\'e}s d'{\'e}nergie suffisamment {\'e}lev{\'e}es, la mati{\`e}re subit une transition de phase vers un {\'e}tat o{\`u} les quarks et les gluons sont lib{\'e}r{\'e}s, le plasma de quarks et de gluons (QGP). Pour recr{\'e}er le QGP en laboratoire, des collisions de noyaux (ions lourds) sont pratiqu{\'e}es dans des acc{\'e}l{\'e}rateurs {\`a} haute {\'e}nergie.

La formation et les caract{\'e}ristiques de QGP dans les collisions noyau-noyau d{\'e}pendent du nombre de nucl{\'e}ons en collision. Pour {\'e}tudier la dynamique de l'environnement nucl{\'e}aire, les collisions d'ions lourds sont class{\'e}es en fonction de leur centralit{\'e}. Exp{\'e}rimentalement, les classes de centralit{\'e} sont d{\'e}finies en mesurant l'{\'e}nergie d{\'e}pos{\'e}e dans des d{\'e}tecteurs souvent positionn{\'e}s vers l'avant. Le nombre moyen de nucl{\'e}ons participant {\`a} la collision ($\left\langle\npart\right\rangle$) et le nombre de collisions binaires nucl{\'e}on-nucl{\'e}on ($\left\langle\ncoll\right\rangle$) sont d{\'e}termin{\'e}s, pour chaque classe de centralit{\'e}, en simulant les collisions de nucl{\'e}ons {\`a} l'aide d'un mod{\`e}le de Monte Carlo (MC) Glauber.

Le QGP ne peut pas {\^e}tre observ{\'e} directement de mani{\`e}re exp{\'e}rimentale, car il n'existe que pour une tr{\`e}s courte p{\'e}riode. N{\'e}anmoins, le QGP peut {\^e}tre {\'e}tudi{\'e} indirectement en mesurant la mani{\`e}re dont les particules (et donc le milieu) produits lors de la collision sont modifi{\'e}s par la pr{\'e}sence du QGP. De nombreuses signatures exp{\'e}rimentales ont {\'e}t{\'e} utilis{\'e}es pour {\'e}valuer les diff{\'e}rentes propri{\'e}t{\'e}s du QGP, telles que l'augmentation de la production de quarks {\'e}tranges, la suppression des quarkonia, l'att{\'e}nuation de l'{\'e}nergie des jets, les anisotropies dans la distribution azimutale des particules (flux elliptique), entre autres. Le m{\'e}canisme de production de chaque sonde exp{\'e}rimentale d{\'e}pend de l'{\'e}chelle du processus. Les signatures produites dans des processus impliquant un fort transfert d'impulsion sont appel{\'e}es sondes dures, tandis que celles produites {\`a} faible transfert d'impulsion sont appel{\'e}es sondes douces.

La majorit{\'e} des particules produites lors de collisions d'ions lourds sont molles. Ces sondes douces sont utilis{\'e}es pour {\'e}tudier l'{\'e}volution thermique et hydrodynamique du milieu. Les rendements de production de particules molles varient progressivement avec \npart. La production des hadrons {\'e}tranges et le flux elliptique sont deux exemples de sondes douces. D'autre part, des sondes dures sont produites {\`a} partir des diffusions dures parton-parton pendant la phase initiale de la collision. Les sondes dures sont des outils id{\'e}aux pour {\'e}tudier la structure du syst{\`e}me car elles sont produites de mani{\`e}re th{\'e}oriquement contr{\^o}l{\'e}e et suffisamment t{\^o}t pour traverser le QGP. Le nombre de particules dures produites est proportionnel {\`a} \ncoll. Certaines des principales sondes dures utilis{\'e}es pour {\'e}tudier le milieu nucl{\'e}aire incluent les bosons {\'e}lectrofaibles, les quarkonia et les jets.

L'environnement pr{\'e}sent dans un noyau peut affecter la production de particules produites lors de collisions d'ions lourds, y compris en l'absence de QGP. La mesure de particules {\'e}lectrofaibles qui n'interagissent pas avec le QGP (photons, bosons Z et W) permet d'{\'e}tudier la modification nucl{\'e}aire des PDF. Les PDF des noyaux sont des informations cruciales pour les pr{\'e}dictions th{\'e}oriques des collisionneurs d'ions lourds{\,;} leur d{\'e}termination pr{\'e}cise {\`a} l'aide de donn{\'e}es exp{\'e}rimentales est indispensable pour quantifier l'{\'e}tat initial des r{\'e}actions noyau-noyau.


\section*{Chapitre 2{~:} Montage exp{\'e}rimental}

Le grand collisionneur de hadrons (LHC) du CERN est actuellement le plus grand et le plus puissant acc{\'e}l{\'e}rateur de particules au monde. Il est install{\'e} dans un tunnel souterrain de {26,7}~\si{\km} de circonf{\'e}rence, situ{\'e} aussi profond que \SI{175}{\m} sous la fronti{\`e}re franco-suisse. Le LHC est capable d'acc{\'e}l{\'e}rer et de faire entrer en collision des faisceaux de protons ou d'ions lourds (par exemple des noyaux de plomb). Les premi{\`e}res collisions noyau-noyau au LHC ont eu lieu en 2010 avec des faisceaux de plomb {\`a} {2,76}~\si{\TeV}. Depuis lors, le LHC est entr{\'e} en collision avec diff{\'e}rentes configurations impliquant des ions, notamment p-Pb {\`a} {2,76}~\si{\TeV} (2013), Pb-Pb {\`a} {5,02}~\si{\TeV} (2015), p-Pb {\`a} {8,16}~\si{\TeV} (2016), Xe-Xe {\`a} {5,44}~\si{\TeV} (2017) et {\`a} la fin de 2018, le LHC pr{\'e}voit de fournir un plus grand ensemble de collisions Pb-Pb {\`a} {5,02}~\si{\TeV}.

Le Compact Muon Solenoid (CMS) est un d{\'e}tecteur de particules polyvalent log{\'e} dans une caverne souterraine au point d'interaction (IP) 5 du LHC. Le d{\'e}tecteur CMS est compos{\'e} d'un tonneau situ{\'e} dans la zone de rapidit{\'e} centrale, ferm{\'e} par deux disques, un de chaque c{\^o}t{\'e} de l'IP, formant un d{\'e}tecteur cylindrique herm{\'e}tique. Il est constitu{\'e} de quatre syst{\`e}mes de sous-d{\'e}tecteurs principaux{~:} le trajectographe en silicium, le calorim{\`e}tre {\'e}lectromagn{\'e}tique (ECAL), le calorim{\`e}tre hadronique (HCAL) et les chambres {\`a} muons. Un aimant sol{\'e}no{\"{i}}dal supraconducteur, plac{\'e} dans le tonneau, engendre un champ magn{\'e}tique de {3,8}~\si{\tesla}. Le trajectographe, l'ECAL et le HCAL sont situ{\'e}s dans le volume de l'{\'e}lectroaimant, tandis que les d{\'e}tecteurs de muons sont plac{\'e}s {\`a} l'ext{\'e}rieur, entre les couches de la culasse {\`a} retour de flux, ce qui limite le flux magn{\'e}tique.

Le syst{\`e}me de trajectographie interne est con{\c{c}}u pour mesurer la trajectoire des particules et reconstruire la position du vertex de l'interaction primaire et des d{\`e}sint{\'e}grations secondaires. Il est compos{\`e} d'un d{\'e}tecteur de pixels et d'un d{\'e}tecteur au silicium {\`a} micropistes. L'ECAL est un calorim{\`e}tre homog{\`e}ne herm{\'e}tique compos{\'e} de tungstate de plomb ($\text{PbWO}_{4}$) et con{\c{c}}u pour mesurer l'{\'e}nergie des {\'e}lectrons et des photons. Le HCAL est un calorim{\`e}tre herm{\'e}tique {\`a} {\'e}chantillonnage constitu{\'e} de dalles en scintillateur plastique, intercal{\'e}es avec des plaques, qui absorbe l'{\'e}nergie des hadrons. Le syst{\`e}me de trajectographie des muons mesure l'impulsion et la charge des muons dans la r{\'e}gion efficace $|\eta|<2,4$, utilisant trois types de technologies gazeuses{~:} les tubes {\`a} d{\'e}rive (DT), les chambres {\`a} pistes cathodiques (CSC) et les chambres {\`a} plaques r{\'e}sistives (RPC).

Dans les conditions de conception du LHC, les deux faisceaux traversent chaque IP toutes les \SI{25}{\ns}, ce qui correspond {\`a} une fr{\'e}quence de \SI{40}{\MHz}. Une fois qu'une collision est enregistr{\'e}e par le d{\'e}tecteur CMS, tous les canaux du d{\'e}tecteur sont lus et les donn{\'e}es sont envoy{\'e}es au centre de calcul du CERN (Tier-0). Toutefois, le d{\'e}bit de traitement de Tier-0 est limit{\'e} par les performances de ses processeurs et sa capacit{\'e} de stockage, et doit {\^e}tre maintenu au-dessous de \SI{1}{\kHz}. Pour atteindre cet objectif, l'exp{\'e}rience CMS a mis en place un syst{\`e}me de d{\'e}clenchement {\`a} deux niveaux. Le premier niveau, appel{\'e} d{\'e}clencheur de niveau 1 (L1), r{\'e}duit le taux de collision {\`a} un taux de sortie de \SI{100}{\kHz}, en filtrant les {\'e}v{\'e}nements {\`a} l'aide de processeurs d{\'e}di{\'e}s. Le niveau de d{\'e}clenchement suivant, appel{\'e} d{\'e}clencheur de haut niveau (HLT), est ex{\'e}cut{\'e} dans un \textit{cluster} d'ordinateurs situ{\'e} dans la caverne du l'exp{\'e}rience CMS. Les algorithmes logiciels du HLT r{\'e}duisent le d{\'e}bit de donn{\'e}es jusqu'{\`a} la limite requise par le Tier-0.

Une fois qu'un {\'e}v{\'e}nement est s{\'e}lectionn{\'e} par le HLT, les informations du d{\'e}tecteur sont transf{\'e}r{\'e}es au centre de calcul Tier-0 et trait{\'e}es dans le cadre logiciel de CMS. Les algorithmes de reconstruction commencent par cr{\'e}er les \textit{hits}, les segments et les clusters mesur{\'e}s dans chacun des sous-d{\'e}tecteurs CMS, et traite ensuite les informations du d{\'e}tecteur pour former des objets physiques tels que des particules charg{\'e}es, des muons, des {\'e}lectrons, des photons et des jets. Les candidats muons sont reconstruits dans CMS en utilisant les informations du syst{\`e}me de trajectographie interne et du syst{\`e}me de muons.

Comme les neutrinos ne peuvent pas {\^e}tre d{\'e}tect{\'e}s, leur pr{\'e}sence est d{\'e}duite du d{\'e}s{\'e}quilibre global d'impulsion des particules dans le plan transverse, connu sous le nom d'impulsion transverse manquante (\ptmiss). La \ptmiss est d{\'e}finie comme la norme de vecteur \ptvecmiss, qui repr{\'e}sente la somme vectorielle inverse de l'impulsion transverse de toutes les particules identifi{\'e}es par le d{\'e}tecteur CMS dans un {\'e}v{\'e}nement.


\section*{Chapitre 3{~:} La production de bosons {\Wb} dans les collisions \RunpPb}

Ce chapitre d{\'e}crit la mesure de la production de bosons {\Wb} dans des collisions proton-plomb {\`a} une {\'e}nergie dans le centre de masse (CM) de nucl{\'e}on-nucl{\'e}on $\sqrtsnn = {8,16}~\si{\TeV}$ avec le d{\'e}tecteur CMS. La production inclusive de bosons {\Wb} est mesur{\'e}e par le canal de d{\'e}croissance muonique, repr{\'e}sent{\'e} par le processus $\pPb\to{\Wb + X}\to{\PGm + \PGnGm + X}$. Puisque la masse du boson {\Wb} est grande  ($M_{\Wb} = {80,385}~\GeVcc$), les bosons {\Wb} se forment pendant les diffusions dures initiales entre les partons du proton entrant et ceux des nucl{\'e}ons li{\'e}s dans l'ion \Pb.

Les bosons {\Wb} sont principalement produits dans les collisions \pPb par les interactions entre les quarks de valence et les anti-quarks de la mer du proton et des nucl{\'e}ons. Le mode de production dominant des bosons {\Wp} correspond {\`a} l'annihilation des quarks up et anti-quarks down ($\cPqu\cPaqd\to\Wp$), tandis que pour des bosons {\Wm} il s'agit de l'annihilation des quarks down avec des anti-quarks up ($\cPqd\cPaqu\to\Wm$). Les contributions suivantes proviennent de $\cPqc\cPaqs$ et $\cPqs\cPaqc$, tandis que les autres contributions quark-antiquark sont supprim{\'e}es en fonction des {\'e}l{\'e}ments non diagonaux de la matrice CKM de couplages des quarks. Par cons{\'e}quent, la section efficace du boson {\Wb} mesur{\'e} dans les donn{\'e}es \RunpPb, est principalement sensible aux PDF nucl{\'e}aires (nPDF) des quarks et des anti-quarks l{\'e}gers.

Dans les collisions d'ions lourds, les PDF des protons et des neutrons li{\'e}s dans un noyau sont modifi{\'e}es par la pr{\'e}sence de l'environnement nucl{\'e}aire. Les PDF sont supprim{\'e}es pour des fractions d'impulsion $x \lesssim 0,1$ (ombrage ou \textit{shadowing}) et augment{\'e}es {\`a} $0,1 \lesssim x \lesssim 0,3$ (\textit{anti-shadowing}) en raison des interactions multiples entre les partons des diff{\'e}rents nucl{\'e}ons. Ils peuvent {\'e}galement {\^e}tre supprim{\'e}s {\`a} $0,3 \lesssim x \lesssim 0,7$ (effet EMC) en raison de modifications de la structure des nucl{\'e}ons et consid{\'e}rablement renforc{\'e}s {\`a} $x > 0,7$ (effet de mouvement de Fermi) r{\'e}sultant du mouvement des nucl{\'e}ons. Les derniers param{\'e}trages des PDF nucl{\'e}aires sont les ensembles EPPS16 et nCTEQ15.

La production de bosons {\Wb} est mesur{\'e}e dans des collisions {\RunpPb} {\`a} l'aide de donn{\'e}es enregistr{\'e}es par le d{\'e}tecteur CMS {\`a} la fin de 2016. L'ensemble de donn{\'e}es utilis{\'e} dans cette analyse est compos{\'e} d'{\'e}v{\'e}nements s{\'e}lectionn{\'e}s par le d{\'e}clencheur HLT, n{\'e}cessitant la pr{\'e}sence d'au moins un candidat muon identifi{\'e} avec $\pt > 12$~\GeVc. La luminosit{\'e} totale int{\'e}gr{\'e}e des donn{\'e}es enregistr{\'e}es correspond {\`a} 173,4~\nbinv, actuellement connue {\`a} 3,5\% pr{\`e}s.

Pendant la p{\'e}riode de prise de donn{\'e}es, les directions des faisceaux de proton et de plomb ont {\'e}t{\'e} permut{\'e}es apr{\`e}s la collecte d'une luminosit{\'e} int{\'e}gr{\'e}e de 62,6~\nbinv. Les {\'e}nergies du faisceau {\'e}taient de {6,5}~\si{\TeV} pour les protons et de {2,56}~\si{\TeV} par nucl{\'e}on pour les noyaux de plomb. Par convention, le c{\^o}t{\'e} vers lequel pointe le proton (\Pb-) d{\'e}finit la r{\'e}gion positive (n{\'e}gative) de pseudo-rapidit{\'e} $\eta$, appel{\'e}e direction avant (arri{\`e}re). En raison du syst{\`e}me de collision asym{\'e}trique, les particules sans masse produites dans le r{\'e}f{\'e}rentiel du centre de masse nucl{\'e}on-nucl{\'e}on {\`a} une pseudo-rapidit{\'e} $\etaCM$ sont reconstruites {\`a} $\etaLAB = \etaCM - 0.465$ dans le cadre de laboratoire. Les mesures du boson {\PWpm} pr{\'e}sent{\'e}es dans cette th{\`e}se sont exprim{\'e}es en termes de la pseudorapidit{\'e} du muon dans le r{\'e}f{\'e}rentiel du CM, $\etaMuCM$.

Les {\'e}v{\'e}nements de signal, d{\'e}termin{\'e}s par le processus \WToMuNupm, sont caract{\'e}ris{\'e}s par la pr{\'e}sence d'un muon isol{\'e} de haut \pt, et d'une importante \ptmiss. Pour am{\'e}liorer la puret{\'e} du signal, la r{\'e}gion efficace de l'analyse a {\'e}t{\'e} limit{\'e}e aux muons de $\pt > 25$~\GeVc avec $\abs{\etaMuLAB} < 2,4$. Les muons sont s{\'e}lectionn{\'e}s en appliquant un crit{\`e}re de s{\'e}lection standard et doivent {\^e}tre isol{\'e}s de l'activit{\'e} hadronique {\`a} proximit{\'e} afin de r{\'e}duire le bruit de fond d{\^u} aux d{\'e}sint{\'e}grations semi-muoniques de hadrons form{\'e}s au sein de jets (appel{\'e} fond de jet QCD). Le param{\`e}tre d'isolation du muon (\iso) est d{\'e}fini comme la somme des \pt de tous les photons, hadrons charg{\'e}s et hadrons neutres, reconstruits dans un c{\^o}ne de rayon $\Delta{R} = 0,3$ autour du candidat muon. Un muon est consid{\'e}r{\'e} isol{\'e} si \iso est inf{\'e}rieur {\`a} 15\% du \pt du muon. Pour supprimer davantage les {\'e}v{\'e}nements d'arri{\`e}re-plan provenant des d{\'e}sint{\'e}grations muoniques de bosons \Z ou de photons virtuels (Drell--Yan), les {\'e}v{\'e}nements contenant au moins deux muons isol{\'e}s de charges oppos{\'e}es, chaque muon ayant $\pt^{\mu} > 15$~\GeVc, sont enlev{\'e}s.

Les sources de fond produisant des muons de haut \pt qui satisfont aux crit{\`e}res de s{\'e}lection de l'analyse sont estim{\'e}es {\`a} l'aide de simulations de Monte Carlo (MC), {\`a} l'exception du fond de jet QCD, d{\'e}crit {\`a} l'aide d'une technique fond{\'e}e sur les donn{\'e}es. La distribution de \ptmiss du fond de jet QCD est mod{\'e}lis{\'e}e par une distribution de Rayleigh modifi{\'e}e qui est param{\'e}tr{\'e}e sur un {\'e}chantillon de muons r{\'e}els et non isol{\'e}s, et extrapol{\'e}e {\`a} la r{\'e}gion de signal de muons isol{\'e}s.

Les {\'e}chantillons simul{\'e}s ont {\'e}t{\'e} engendr{\'e}s au NLO {\`a} l'aide du g{\'e}n{\'e}rateur \POWHEG v2. La simulation des collisions \pPb est effectu{\'e}e {\`a} l'aide de l'ensemble de PDF CT14 corrig{\'e} par les facteurs de modification nucl{\'e}aire EPPS16. Les densit{\'e}s de partons des protons et des neutrons sont mises {\`a} l'{\'e}chelle en fonction de la masse et du num{\'e}ro atomique des isotopes de plomb. Les gerbes partoniques sont simul{\'e}es en hadronisant les {\'e}v{\'e}nements \POWHEG avec \PYTHIA 8.212, {\`a} l'aide de l'ajustement d'{\'e}v{\'e}nement sous-jacent CUETP8M1. La r{\'e}ponse compl{\`e}te du d{\'e}tecteur CMS est simul{\'e}e dans tous les {\'e}chantillons MC, sur la base de \GEANTfour, en consid{\'e}rant un alignement et un {\'e}talonnage r{\'e}alistes des diff{\'e}rents sous-d{\'e}tecteurs. Pour envisager une distribution plus r{\'e}aliste de l'environnement sous-jacent pr{\'e}sent dans les collisions \RunpPb, les {\'e}v{\'e}nements de signal MC ont {\'e}t{\'e} incorpor{\'e}s dans un {\'e}chantillon {\`a} biais minimal (i.e. interactions in{\'e}lastiques hadroniques) g{\'e}n{\'e}r{\'e} avec \EPOSLHC, en tenant compte des deux directions de collisions \RunpPb.

Afin d'am{\'e}liorer l'accord entre les simulations {\'e}lectrofaibles et les donn{\'e}es, la distribution du \pt du boson faible est pond{\'e}r{\'e}e {\`a} l'aide d'une fonction d{\'e}pendant de \pt d{\'e}riv{\'e}e du rapport des distributions du \pt du boson {\PZ} dans les {\'e}v{\'e}nements \ZToMuMu des donn{\'e}es et de la simulation. De plus, les {\'e}v{\'e}nements \pPb sont pond{\'e}r{\'e}s en faisant correspondre la distribution d'{\'e}nergie simul{\'e}e reconstruite dans les calorim{\`e}tres hadroniques avant {\`a} celle observ{\'e}e dans les donn{\'e}es d'un {\'e}chantillon \DYToMuMu. Enfin, le recul simul{\'e} des bosons {\PW} et {\PZ}, d{\'e}fini comme la somme vectorielle des \pt de toutes les particules reconstruites {\`a} l'exclusion des produits de d{\'e}sint{\'e}gration du boson faible, est calibr{\'e} de mani{\`e}re {\`a} ce que sa distribution moyenne corresponde {\`a} celle des donn{\'e}es.

Le nombre d'{\'e}v{\'e}nements de signal \WToMuNu est obtenu en effectuant un ajustement de vraisemblance maximale de la distribution \ptmiss observ{\'e}e dans diff{\'e}rentes r{\'e}gions de  \etaMuCM. Le mod{\`e}le d'ajustement total comprend six contributions{~:} le mod{\`e}le du signal \WToMuNu, les mod{\`e}les des processus des fonds \DYToMuMu, \WToTauNu, \DYToTauTau et \ttbar, et la forme fonctionnelle du fond de jet QCD.

Les sections efficaces pour les d{\'e}sint{\'e}grations \WToMuNupm mesur{\'e}es dans les collisions {\RunpPb} {\`a} {8,16}~\si{\TeV} sont compar{\'e}es aux calculs PDF NLO utilisant les PDF des nucl{\'e}ons CT14, y compris les modifications nucl{\'e}aires donn{\'e}es par les ensembles EPP16 et nCTEQ15. Les mesures {\`a} rapidit{\'e} positive  favorisent les calculs qui incluent les modifications nucl{\'e}aires, tandis que les trois calculs {\`a} rapidit{\'e} n{\'e}gative sont en bon accord avec les donn{\'e}es.

Les taux des muons charg{\'e}s positivement et n{\'e}gativement ($N_{\PGm}$) sont ensuite combin{\'e}s pour mesurer les rapports avant-arri{\`e}re $N _ {\PGm}(+\etaMuCM)/N_{\PGm}(-\etaMuCM)$. Les r{\'e}sultats sont en bon accord avec les calculs de PDF nucl{\'e}aire EPPS16 et nCTEQ15. Par ailleurs, les mesures de boson {\PW} contredisent de mani{\`e}re significative les calculs fond{\'e}s sur les PDF de nucl{\'e}ons nus CT14, r{\'e}v{\'e}lant sans ambigu{\"i}t{\'e} la pr{\'e}sence de modifications nucl{\'e}aires dans la production de bosons {\'e}lectrofaibles, pour la premi{\`e}re fois. Compte tenu de la taille des incertitudes mesur{\'e}es, plus petite que celles du mod{\`e}le, les mesures de boson {\PW} imposent des contraintes fortes sur les param{\'e}trages des PDF nucl{\'e}aires des quarks et des anti-quarks.


\section*{Chapitre 4{~:} Production des charmonia dans les collisions {\RunPbPb}}

Les charmonia sont des {\'e}tats li{\'e}s d'un quark charm et d'un anti-quark charm. Les charmonia peuvent {\^e}tre produits {\`a} partir de diverses sources, notamment{~:} la diffusion dure initiale (directe), les d{\'e}sint{\'e}grations d'{\'e}tats charmonium de masse sup{\'e}rieure (feed-down) ou les d{\'e}sint{\'e}grations faibles de hadrons contenant des quarks bottom. Les {\'e}tats charmonium produits directement ou provenant de contributions de feed-down sont appel{\'e}s \textit{prompt}, alors que les {\'e}tats charmonium issus de d{\'e}sint{\'e}grations de hadron b sont appel{\'e}s \textit{nonprompt}.

Les taux des charmonia observ{\'e}s sont modifi{\'e}s dans les collisions d'ions lourds par un jeu d'effets diff{\'e}rents pouvant se produire dans l'{\'e}tat initial ou final de la collision. Les effets provenant de l'environnement nucl{\'e}aire sont souvent appel{\'e}s effets de la mati{\`e}re nucl{\'e}aire froide (CNM), tandis que ceux caus{\'e}s par le milieu chaud et dense form{\'e} lors de la collision, le QGP, sont appel{\'e}s effets de la mati{\`e}re nucl{\'e}aire chaude (HNM). La compr{\'e}hension de l'impact des effets de la mati{\`e}re nucl{\'e}aire froide est cruciale pour pouvoir caract{\'e}riser le milieu chaud produit lors de collisions d'ions lourds. La production des charmonia peut {\^e}tre affect{\'e}e par plusieurs effets CNM, tels que l'absorption nucl{\'e}aire, l'ombrage des gluons, la perte d'{\'e}nergie et l'effet Cronin.

Les charmonia sont consid{\'e}r{\'e}s comme des sondes importantes du QGP car ils sont produits lors de la diffusion dure initiale et subissent toute l'{\'e}volution du milieu. On s'attend {\`a} ce que la pr{\'e}sence du milieu d{\'e}confin{\'e} dissocie les {\'e}tats du charmonium gr{\^a}ce {\`a} un processus appel{\'e} {\'e}crantage de la charge de couleur, qui peut se produire de mani{\`e}re s{\'e}quentielle en fonction des {\'e}nergies de liaison du charmonium. De plus, la grande abondance de quarks charm au LHC peut conduire {\`a} une recombinaison de quarks charm non corr{\'e}l{\'e}s, augmentant ainsi la production des charmonia.

La production de m{\'e}sons \JPsi a {\'e}t{\'e} mesur{\'e}e au LHC lors de collisions {\RunPbPb} {\`a} $\sqrtsnn = {2,76}~\si{\TeV}$. La caract{\'e}ristique g{\'e}n{\'e}rale observ{\'e}e parmi les diff{\'e}rentes exp{\'e}riences du LHC est une forte suppression du charmonia lors de collisions centrales en coh{\'e}rence avec l'{\'e}crantage de la charge de couleur. De plus, la collaboration ALICE a signal{\'e} une suppression plus faible des m{\'e}sons \JPsi, en particulier {\`a} faible \pt, par rapport aux mesures au RHIC, ce qui a {\'e}t{\'e} attribu{\'e}e {\`a} la r{\'e}g{\'e}n{\'e}ration du m{\'e}son \JPsi. Des mesures dans des collisions \RunpPb ont {\'e}galement {\'e}t{\'e} effectu{\'e}es au LHC, qui se sont r{\'e}v{\'e}l{\'e}es coh{\'e}rentes avec les calculs incluant les modifications nucl{\'e}aires des PDF et/ou les pertes d'{\'e}nergie. Cependant, les contributions exactes des divers effets de la mati{\`e}re nucl{\'e}aire {\`a} chaud et {\`a} froid sont difficiles {\`a} {\'e}valuer, en particulier en raison des grandes incertitudes sur les PDF nucl{\'e}aires du gluon et de la pr{\'e}cision statistique limit{\'e}e des donn{\'e}es. En cons{\'e}quence, des mesures plus pr{\'e}cises et diff{\'e}rentielles sont n{\'e}cessaires, {\`a} la fois pour contraindre les mod{\`e}les et pour d{\'e}m{\^e}ler les diff{\'e}rentes contributions qui jouent un r{\^o}le dans les collisions d'ions lourds.

Dans ce chapitre, deux analyses associ{\'e}es de la production des charmonia dans les collisions \Runpp et {\RunPbPb} {\`a} $\sqrtsnn = {5,02}~\si{\TeV}$ sont d{\'e}crites. La premi{\`e}re analyse {\'e}tudie la modification de la production prompte et non-prompte des m{\'e}sons \JPsi dans \RunPbPb par rapport aux collisions de {\Runpp} {\`a} la m{\^e}me {\'e}nergie. Pour ce faire, le facteur de modification nucl{\'e}aire (\raa) des m{\'e}sons \JPsi est mesur{\'e} dans diff{\'e}rentes classes de centralit{\'e} de collision et intervalles de cin{\'e}matique du m{\'e}son \JPsi. La deuxi{\`e}me analyse porte sur la modification nucl{\'e}aire des m{\'e}sons \PsiP par rapport aux m{\'e}sons \JPsi, en mesurant le double rapport des taux de \PsiP sur \JPsi dans \RunPbPb par rapport aux collisions \Runpp, d{\'e}finies comme \doubleRatio.

Les candidats charmonium sont reconstruits dans le canal de d{\'e}sint{\'e}gration en deux muons (i.e. \JPsiToMuMu et \PsiPToMuMu), en appariant des muons de charge oppos{\'e}e. Puisque les masses \JPsi et \PsiP sont petites ($\text{m}_{\JPsi} = 3,097~\GeVcc$ et $\text{m}_{\PsiP} = 3,686~\GeVcc$), les {\'e}v{\'e}nements de signal sont domin{\'e}s par la pr{\'e}sence de muons de faible \pt ($\langle\pt^{\PGm}\rangle \sim 1.6~\GeVc$), contrairement {\`a} l'analyse des bosons {\PW} pr{\'e}sent{\'e}e au chapitre~3. Les {\'e}v{\'e}nements de fond sont supprim{\'e}s en exigeant que chaque candidat double muon ait une probabilit{\'e} $\chi^2$ sup{\'e}rieure {\`a} $1\%$ que les deux muons d{\'e}rivent d'un vertex commun.

Les taux des m{\'e}sons \JPsi prompts et non prompts sont extraits en effectuant un ajustement de vraisemblance maximum {\`a} deux dimensions des distributions de masse invariante de \mumu (\mMuMu) et de longueur de la d{\'e}sint{\'e}gration de pseudo-proper (\ctau). Le \ctau des candidats \mumu est d{\'e}finie comme $\ctau = \mJPsi\cdot({\pMuMuvec\cdot\vec{r}})/({\left(\pMuMu\right)^{2}})$, o{\`u} $\mJPsi = 3.0969~\GeVcc$ est la masse du m{\'e}son \JPsi, $\pMuMuvec$ est le vecteur d'impulsion du double muon et $\vec{r}$ est le vecteur de d{\'e}placement entre la position du vertex de collision principal et le vertex du double muon.

Pour extraire les taux des m{\'e}sons \PsiP prompts, les doubles muons doivent passer une s{\'e}lection sur \ctau qui rejette les dimuons avec des valeurs de \ctau sup{\'e}rieures {\`a} un seuil donn{\'e}. Le seuil de s{\'e}lection sur \ctau est optimis{\'e} {\`a} l'aide de simulations, en conservant 90\% des charmonia prompts tout en rejetant plus de 80\% des charmonia non prompts. Le rapport des taux des m{\'e}sons \PsiP sur \JPsi est extrait des donn{\'e}es en ajustant la distribution du \mMuMu des doubles muons passant la s{\'e}lection sur \ctau, et le rapport des taux des m{\'e}sons \PsiP prompt sur \JPsi prompt est d{\'e}termin{\'e} par soustraction du charmonia non prompts qui passe la s{\'e}lection sur \ctau.

La production de m{\'e}sons \JPsi prompts et non prompts s'av{\`e}re supprim{\'e}e dans toutes les mesures. On observe que le facteur de modification nucl{\'e}aire \raa des m{\'e}sons \JPsi d{\'e}pend de la centralit{\'e}, {\'e}tant davantage supprim{\'e} pour les collisions plus centrales, alors qu'aucune d{\'e}pendance significative en la rapidit{\'e} n'est observ{\'e}e. D'une part, on observe une indication de suppression plus faible pour les m{\'e}sons \JPsi prompts, dans l'intervalle d'impulsion transverse le plus bas ($3 < \pt < 6.5$~\GeVc) et la plupart des collisions centrales (0-30\%), pouvant provenir de la r{\'e}g{\'e}n{\'e}ration des m{\'e}sons \JPsi. En revanche, la suppression des m{\'e}sons \JPsi non prompts semble {\^e}tre plus prononc{\'e}e {\`a} haute \pt, probablement {\`a} cause de l'att{\'e}nuation des jets (jet \textit{quenching}) des quarks bottom. Dans la plage de chevauchement, le \raa des m{\'e}sons \JPsi mesur{\'e} est compatible avec les mesures pr{\'e}c{\'e}dentes {\`a} $\sqrtsnn = {2,76}~\si{\TeV}$.

Les r{\'e}sultats des \raa des charmonia prompts sont compatibles {\`a} ceux des m{\'e}sons $\text{D}^{0}$, en particulier {\`a} $\pt > 15$~\GeVc, o{\`u} \raa augmente avec \pt. Cette augmentation est comprise dans le contexte de l'att{\'e}nuation des jets. Ceci sugg{\`e}re que la perte d'{\'e}nergie devrait jouer un r{\^o}le important pour les m{\'e}sons \JPsi, comme pour tout autre hadron. Les m{\'e}sons \JPsi prompts {\`a} haut \pt doivent provenir en partie de la fragmentation des gluons ou des {\'e}tats de quarkonium de couleur-octet, qui sont soumis aux effets de perte d'{\'e}nergie induits par le milieu QGP.

La mesure du double rapport \doubleRatio dans {\RunPbPb} {\`a} $\sqrtsnn = {5,02}~\si{\TeV}$ montre que les m{\'e}sons \PsiP sont plus supprim{\'e}s que les m{\'e}sons \JPsi, ce qui est compatible avec le sc{\'e}nario de suppression s{\'e}quentielle des charmonia dans le QGP. Les comparaisons avec les mesures {\`a} $\sqrtsnn = {2,76}~\si{\TeV}$ montrent un bon accord du double rapport {\`a} haut \pt dans la r{\'e}gion de mid-rapidit{\'e}. Au contraire, l'extension de la plage de \pt jusqu'{\`a} 3~\GeVc dans la r{\'e}gion de rapidit{\'e} vers l'avant montre une r{\'e}duction plus importante du double rapport {\`a} $\sqrtsnn = {5,02}~\si{\TeV}$ par rapport {\`a} $\sqrtsnn = {2,76}~\si{\TeV}$. Une r{\'e}g{\'e}n{\'e}ration s{\'e}quentielle du charmonia a {\'e}t{\'e} sugg{\'e}r{\'e}e pour expliquer les r{\'e}sultats du double rapport aux deux {\'e}nergies de collision.
