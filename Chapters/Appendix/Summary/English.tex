\chapter{Summary in English}\label{app:Summary_English}


\section{Chapter 1: High energy nuclear physics}

The progress made by the scientific community over the last century has pushed the boundaries of our understanding of the subatomic world and led to the formulation of one the most successful theories of physics, corresponding to the Standard Model (SM) of particle physics. The SM is a theoretical framework that describes the properties of elementary particles and their interactions. According to the SM, the most elementary particles are fermions and bosons.

Fermions are particles with half-integer spin which behave according to Fermi-Dirac statistics and are classified in two categories: leptons and quarks. There are six leptons arranged in three "generations": the electron (\PGem) and the electron neutrino (\PGnGe), the muon (\PGmm) and the muon neutrino (\PGnGm), and the tau (\PGtm) and the tau neutrino (\PGnGt). In the case of quarks, there are six "flavours" paired also in three generations: the up (\cPqu) and the down (\cPqd) quarks, the charm (\cPqc) and the strange (\cPqs) quarks, and the top (\cPqt) and the bottom (\cPqb) quarks.

The interactions between fermions are described in the SM by three fundamental forces: the electromagnetic force, the strong nuclear force and the weak nuclear force. Each fundamental force is mediated by the exchange of bosons, which are integer spin particles that follows the Bose-Einstein statistics. The electromagnetic interactions between particles with electric charge are mediated by photons. The weak interactions can act on all fermions and are mediated by three vector bosons: the \Wpm and {\PZ} bosons. The strong interactions are mediated by gluons and are described by the theory of Quantum Chromodynamics (QCD).

Quantum Chromodynamics is a non-abelian quantum field theory with gauge symmetry group SU(3). The primary objects of QCD are the quarks which carry one colour charge and the gluons which carry a colour and an anti-colour charge. The strength of the strong interactions is parametrised by the running coupling constant $\alpha_{s}$, which depends on the energy scale $Q$. At lower energies the strong coupling  increases, confining the partons (i.e. quarks and gluons) into hadrons, while at higher energies the strong coupling decreases, leading to the asymptotic freedom of partons. Hadrons composed of three (anti-)quarks are called (anti-)baryons while those made of a quark and an anti-quark are called mesons.

The production of particles in hadronic collisions depends on the evolution of the partons inside the hadrons. The partonic content of hadrons can be studied through the parton distribution functions (PDF), which represent the probability that a parton carries a given fraction $x$ (also called Bjorken $x$) of the total momentum of the hadron. According to the QCD factorisation theorem, the cross section of a given hard scattering process in hadronic collisions can be split in a partonic cross section times the PDFs of each incoming hadron. On one hand, the partonic cross section can be derived using perturbative QCD and does not depend on the colliding hadrons. On the other hand, the  PDFs can not be calculated from first principles due to the non-perturbative nature of QCD, but they can be determined from global fits to experimental data since the PDFs are independent of the initial scattering process.

Normal nuclear matter exists in nature at low temperatures and relatively high baryon densities. However, at high enough temperatures or energy densities, matter undergoes a phase transition to a state where quarks and gluons become asymptotically free, known as the Quark Gluon Plasma (QGP). To recreate the QGP in the laboratory, heavy ions are collided in accelerator facilities at high energies. 

The formation and characteristics of the QGP in nucleus-nucleus collisions depends on the number of colliding nucleons. To study the dynamics of the nuclear medium, the heavy-ion collisions are classified based on their centrality. Experimentally, the centrality classes are defined by binning the energy distribution measured in the forward calorimeters, so that each bin contains the same fraction of the total integral. The mean number of partons that participate in the collision ($\left\langle\npart\right\rangle$) and number of binary partonic collisions ($\left\langle\ncoll\right\rangle$) are determined, for each centrality class, by simulating the charged-particle distribution using a Monte Carlo (MC) Glauber model.

The QGP can not be directly measured experimentally, since it only exists for a very short amount of time. Nonetheless, the QGP can be studied indirectly by measuring how the particles and the system produced in the collision are modified by the presence of the QGP. There are many experimental \textit{signatures} that have been used to assess the different properties of the QGP, such as the enhancement of the strange quark production, suppression of the quarkonium yields, attenuation of the energy of jets, anisotropies in the azimuthal distribution of particles, among others. The production mechanism of each experimental probe depends on the momentum scale of the process. Signatures produced in processes involving large momentum transfer are called hard probes while those produced at low momentum scales are called soft probes.

The majority of the particles produced in heavy-ion collisions are soft and constitutes the bulk of the system. Soft probes are used to study the thermal and hydrodynamical evolution of the medium. The production yields of soft particles scale with \npart. The strange hadron yields and the elliptic flow are two examples of soft probes. On the other hand, hard probes are produced from the parton-parton hard scatterings during the initial stage of the collision. Hard probes are ideal tools to study the structure of the system since they are produced early in a well-controlled manner and often living through the QGP. The number of hard particles produced in the medium scales with \ncoll. Some important hard probes used to study the nuclear medium includes the electroweak bosons, quarkonia and jets. 

The environment present in a nucleus can affect the production of particles produced in heavy-ion collisions, even in the absence of QGP. The measurement of electroweak particles that do not interact with the QGP medium (photons, Z and W bosons) allows to study the nuclear modification of PDFs. The PDFs of nuclei are crucial inputs to theory predictions for heavy-ion colliders and their precise determination with experimental data is indispensable for calculations of the initial stage of nucleus-nucleus reactions.


\section{Chapter 2: Experimental setup}

The Large Hadron Collider at CERN is currently the largest and highest-energy particle accelerator in the world. It is installed in an underground tunnel of \SI{26.7}{\km} in circumference, located as deep as \SI{175}{\m} underground beneath the border between France and Switzerland. The LHC is capable of accelerating and colliding beams of protons or heavy ions (e.g Pb nuclei). The first nucleus-nucleus collisions at LHC took place in 2010 using Pb beams at \SI{2.76}{\TeV}. Since then, the LHC has collided different configurations involving ions, including proton-Pb at \SI{2.76}{\TeV} (2013), Pb-Pb at 5.02~TeV (2015), proton-Pb at \SI{8.16}{\TeV} (2016), Xenon-Xenon at \SI{5.44}{\TeV} (2017), and at the end of 2018 LHC is planning to provide a larger set of Pb-Pb collisions at \SI{5.02}{\TeV}.

The Compact Muon Solenoid (CMS) experiment is a multi-purpose particle detector housed in an underground cavern at the interaction point (IP) 5 of the LHC. The CMS is composed of a central barrel in the mid-rapidity region closed by two endcap disks, one on each side of the IP, forming a hermetic cylindrical detector. It is made of four main subdetector systems: the tracker, the Electromagnetic CALorimeter (ECAL), the Hadronic CALorimeter (HCAL) and the muon chambers. A superconducting solenoid magnet, placed in the barrel section, generates a magnetic field of \SI{3.8}{\tesla}. The tracking system, the ECAL and the HCAL, are located within the solenoid volume, while the muon system is placed between the layers of the flux-return yoke, which confines the magnetic flux.

The inner tracking system is designed to measure the trajectory of charged particles and reconstruct the  vertex position of the primary interaction and the secondary decays. It is made of a pixel detector and a silicon strip tracker. The ECAL is a hermetic homogeneous calorimeter composed of lead-tungstate ($\text{PbWO}_{4}$) crystals and designed to absorb and measure the energy of electrons and photons. The HCAL is a hermetic sampling calorimeter made of plastic-scintillator tiles interleaved with absorber plates, which absorbs and measures the energy of hadrons. The muon tracking system measures the momentum and charge of muons in the fiducial region $|\eta| < 2.4$, using three type of gaseous technologies: Drift Tubes (DT), Cathode Strip Chambers (CSC) and Resistive Plate Chambers (RPC).

At LHC design conditions, the two beams cross each IP every \SI{25}{\ns}, equivalent to a frequency of \SI{40}{\MHz}. Once a collision is recorded by CMS, all detector channels are read out and the data is sent to the CERN main computing farm (Tier-0). However, the Tier-0 processing rate is limited by its CPU performance and storage capacity, and has to be kept below \SI{1}{\kHz}. To reach this goal, CMS has implemented a two-level trigger system designed to select events of interest for physics analysis. The first level, known as the Level-1 (L1) trigger, lowers the collision rate to an output rate of \SI{100}{\kHz} by filtering events using custom hardware. The next trigger level, called the High Level Trigger (HLT), is performed in a cluster of computers located in the CMS experimental cavern. The HLT software algorithms further reduce the data rate down the limit required by the Tier-0.

Once an event is selected by the HLT, the detector information is transferred to the Tier-0 computing centre and processed with the CMS software framework. The reconstruction algorithms starts by building the hits, segments and clusters, measured in each of the CMS subdetectors, and then process the detector information to form physics objects such as charged-particle tracks, muons, electrons, photons and jets. The muon candidates are reconstructed in CMS using the information from the inner tracker and the muon system.

Since neutrinos cannot be detected, their presence is inferred from the overall particle momentum imbalance in the transverse plane, known as missing transverse momentum (\ptmiss). The \ptmiss is defined as the magnitude of \ptvecmiss, which represents the negative vector sum of the transverse momentum of all particles identified by CMS in an event.


\section{Chapter 3: \Wb-boson production in proton-lead collisions}

This chapter reports the measurement of the production of \Wb bosons in proton-lead collisions at a nucleon-nucleon center-of-mass energy $\sqrtsnn = \SI{8.16}{\TeV}$ with the CMS detector. The inclusive production of \Wb bosons is measured through the muonic decay channel, which is represented by the process $\pPb\to{\Wb + X}\to{\PGm + \PGnGm + X}$. Since the mass of the \Wb boson is large ($M_{\Wb} = \SI{80.385}{\GeV}$), the \Wb bosons are formed during the initial hard scatterings between the partons from the incoming proton and those from the nucleons bound in the \Pb ion.

\Wb bosons are mainly produced in \pPb collisions from interactions between the valence quarks and sea anti-quarks of the proton and nucleons. The dominant production mode of \Wp bosons corresponds to up quark and down anti-quark annihilation ($\cPqu\cPaqd\to\Wp$), while for \Wm bosons is the annihilation of down quarks with up anti-quarks ($\cPqd\cPaqu\to\Wm$). The next relevant contributions come from $\cPqc\cPaqs$ and $\cPqs\cPaqc$, while the other quark-antiquark contributions are suppressed according to the off-diagonal CKM matrix elements. Thus, the inclusive \Wb boson cross section measured in \RunpPb data is mostly sensitive to the nuclear PDFs (nPDFs) of light quarks and anti-quarks.

In heavy-ion collisions, the PDFs of protons and neutrons bound in a nucleus are modified by the presence of the nuclear environment. The PDFs are suppressed at $x \lesssim 0.1$ (shadowing) and enhanced at $0.1 \lesssim x \lesssim 0.3$ (anti-shadowing) due to multiple interactions between the scattered partons and the ones from the different nucleons. They can also be suppressed at $0.3 \lesssim x \lesssim 0.7$ (EMC effect) due to modifications of the nucleon structure and significantly enhanced at $x > 0.7$ (Fermi motion effect) arising from the motion of nucleons. The latest parametrisations of the nuclear PDFs are the EPPS16 and nCTEQ15 nPDF sets.

The production of \Wb bosons is measured in \RunpPb collisions using data recorded by the CMS detector at the end of 2016. The dataset employed in this analysis is composed of events selected by the HLT trigger, requiring the presence of at least one identified muon candidate with $\pt > 12$~\GeVc. The total integrated luminosity of the recorded data corresponds to 173.4~\nbinv, currently known within 3.5\%.

During the data-taking period, the directions of the proton and lead beams were swapped after an integrated luminosity of 62.6~\nbinv was collected. The beam energies were \SI{6.5}{\TeV} for the protons and \SI{2.56}{\TeV} per nucleon for the lead nuclei. By convention, the proton-(\Pb-)going side defines the positive (negative) $\eta$ region, labelled as the forward (backward) direction. Due to the asymmetric collision system, massless particles produced in the nucleon-nucleon center-of-mass frame at an $\etaCM$ are reconstructed at $\etaLAB = \etaCM - 0.465$ in the laboratory frame. The {\PWpm} boson measurements presented in this thesis are expressed in terms of the muon pseudorapidity in the CM frame, $\etaMuCM$.

The signal events, determined by the process \WToMuNupm, are characterised by the presence of an isolated high-\pt muon and \ptmiss. To enhance the signal purity, the fiducial region of the analysis has been restricted to muons of $\pt > 25$~\GeVc with $\abs{\etaMuLAB} < 2.4$. Muons are selected by applying a standard selection criteria and are required to be isolated from nearby hadronic activity to reduce the background from semi-muonic decays of hadrons formed within jets (referred as QCD jet background). The muon isolation parameter ($I_{\mu}$) is defined as the \pt sum of all reconstructed photons, charged and neutral hadrons, in a cone of radius $\Delta{R} = 0.3$ around the muon candidate in the $\eta$-$\phi$ plane. A muon is considered isolated if $I_{\mu}$ is less than 15\% of the muon \pt. To further suppress background events from dimuon decays of \Z bosons or virtual photons (Drell--Yan), events containing at least two isolated oppositely charged muons, each with $\pt^{\mu} > 15$\GeVc, are removed.

Background sources yielding high-\pt muons that pass the analysis selection criteria are estimated using Monte Carlo (MC) simulations, with the exception of the QCD jet background, which is described using a  data-driven technique. The \ptmiss shape of the QCD jet background is modelled by a modified Rayleigh distribution parametrised in a sample of non-isolated muons in data and extrapolated to the isolated muon signal region.

The simulated samples were generated at NLO using the \POWHEG v2 generator. The simulation of \pPb collisions is performed using the CT14 PDF set corrected with the EPPS16 nuclear modification factors derived for Pb ions. The parton densities of protons and neutrons are scaled according to the mass and atomic number of the lead isotopes. The parton showering is performed by hadronizing the \POWHEG events with \PYTHIA 8.212, using the CUETP8M1 underlying event tune. The full CMS detector response is simulated in all MC samples, based on \GEANTfour, considering a realistic alignment and calibration of the different sub-detectors of CMS. To consider a more realistic distribution of the underlying environment present in \RunpPb collisions, the MC signal events were embedded in a minimum bias (i.e. inelastic hadronic interactions) sample generated with \EPOSLHC, taking into account both \RunpPb boost directions.

In order to improve the agreement between the electroweak simulations and the data, the weak boson \pt distribution is weighed using a \pt-dependent function derived from the ratio of the {\PZ} boson \pt distributions in \ZToMuMu events in data and simulation. Furthermore, the \pPb event activity is weighed by matching the simulated energy distribution reconstructed in the forward hadronic calorimeters to the one observed in data in a \DYToMuMu sample. Finally, the simulated recoil of {\PW} and {\PZ} bosons, defined as the vector \pt sum of all reconstructed particles excluding the decay products of the weak boson, is calibrated so that its average distribution matches the one in data.

The number of \WToMuNu signal events is obtained by performing an unbinned maximum-likelihood fit of the observed \ptmiss distribution in different muon \etaMuCM regions. The total fit model includes six contributions: the signal \WToMuNu template, the background templates from the \DYToMuMu, \WToTauNu,  \DYToTauTau and \ttbar processes, and the QCD jet background functional form.

The cross sections for the \WToMuNupm decays measured in \RunpPb collisions at \SI{8.16}{\TeV} are compared to  NLO PDF calculations using the CT14 proton PDF including nuclear modifications based on the EPPS16 and nCTEQ15 nPDF sets. The measurements at forward rapidity favour the PDF calculations including nuclear modifications, while at backward rapidity all three PDF calculations are in good agreement with the data.

The yields of positively and negatively charged muons ($N_{\PGm}$) are further combined to measure  forward-backward ratios $N_{\PGm}(+\etaMuCM)/N_{\PGm}(-\etaMuCM)$. The results are found to be in good agreement with the EPPS16 and nCTEQ15 nuclear PDF calculations. On the other hand, the \PW-boson measurements significantly disfavour the CT14 proton PDF calculations, revealing unambiguously the presence of nuclear modifications in the production of electroweak bosons, for the first time. Considering the smaller size of the measured uncertainties, compared to the model calculations, the \PW-boson measurements have the  potential to constrain the parametrisations of the (anti-)quark nuclear PDFs.


\section{Chapter 4: Charmonium production in lead-lead collisions}

Charmonia are bound states of a charm quark and anti-quark. Charmonia can be produced from various sources including: the initial hard scattering (direct), decays of higher mass charmonium states (feed-down), or weak decays of hadrons containing bottom quarks. Directly produced charmonium states or those from feed-down contributions are known as \textit{prompt}, while charmonium states from b-hadron decays are called \textit{nonprompt}.

The observed yields of charmonia are modified in heavy-ion collisions by an interplay of different effects that can take place in the initial or final state of the collision. The effects that originate from the nuclear environment are often called cold nuclear matter (CNM) effects, while those that are caused by the hot and dense medium formed in the collision, the QGP, are known as hot nuclear matter (HNM) effects. Understanding the impact of the cold nuclear matter effects is crucial to be able to characterise the hot medium produced in heavy-ion collisions. The charmonium production can be affected by several CNM effects, such as nuclear absorption, gluon shadowing, energy loss and Cronin effect.

Charmonia are considered important probes of the QGP since they are produced in the initial hard scattering and experience the full evolution of the medium. The presence of the deconfined medium is expected to dissociate the charmonium states through a process called colour-charge screening, which can occur sequentially depending on the medium temperature and the charmonium binding energies. In addition, the large abundance of charm quarks at the LHC can lead to a recombination of uncorrelated charm quarks, enhancing the charmonium yields.

The production of \JPsi mesons has been measured at the LHC in \RunPbPb collisions at $\sqrtsnn = \SI{2.76}{\TeV}$. The general feature observed among the different LHC experiments is a strong suppression of charmonia in central collisions consistent with colour-charge screening. In addition, the ALICE collaboration has reported a weaker suppression of \JPsi mesons in particular at low \pt compared to RHIC measurements, which has been attributed to \JPsi-meson regeneration. Measurements in \RunpPb collisions have also been performed at the LHC, which are found to be consistent with calculations including nuclear modifications of the PDFs and/or energy loss. However, the exact contributions of the various hot and cold nuclear matter effects are difficult to asses, specially due to the large uncertainties on the gluon nuclear PDFs and the limited statistical precision of the data. As a result, more precise and differential measurements are needed, both to constrain the models and to disentangle the different contributions that play a role in heavy-ion collisions.

In this chapter, two related analyses of the charmonium production in \Runpp and \RunPbPb collisions at $\sqrtsnn = \SI{5.02}{\TeV}$ are described. The first analysis studies the modification of the prompt and nonprompt \JPsi meson production in \RunPbPb compared to \Runpp collisions at the same energy. To accomplish this, the nuclear modification factor (\raa) of \JPsi mesons is measured in different collision centrality bins, and \JPsi-meson kinematic ranges. The second analysis probes the nuclear modification of \PsiP mesons relative to \JPsi mesons, by measuring the double ratio of \PsiP over \JPsi yields in \RunPbPb relative to \Runpp collisions, defined as \doubleRatio.

The charmonium candidates are reconstructed in the dimuon decay channel (i.e. \JPsiToMuMu and \PsiPToMuMu), by pairing opposite-charge muons. Since the \JPsi and \PsiP masses are small ($\text{m}_{\JPsi} = 3.097~\GeVcc$ and $\text{m}_{\PsiP} = 3.686~\GeVcc$), the signal events are dominated by the presence of low \pt muons ($\langle\pt^{\PGm}\rangle \sim 1.6~\GeVc$), contrary to the \PW-boson analysis reported in \chp{sec:WBoson}. Background events are suppressed by requiring each dimuon candidate to have a $\chi^2$ probability larger than $1\%$ that both muons derive from a common vertex.

The prompt and nonprompt \JPsi-meson yields are extracted by performing a two-dimensional unbinned-maximum likelihood fit to the \mumu invariant mass (\mMuMu) and pseudoproper-decay length (\ctau) distributions. The \ctau of \mumu candidates, used to estimate the b-hadron decay length, is defined as $\ctau = \mJPsi\cdot({\pMuMuvec \cdot \vec{r}})/({\left(\pMuMu\right)^{2}})$, where $\mJPsi = 3.0969~\GeVcc$ is the mass of the \JPsi meson, $\pMuMuvec$ is the dimuon momentum vector and $\vec{r}$ is the displacement vector between the position of the primary collision vertex and the dimuon vertex.

In order to extract the yields of prompt \PsiP mesons, the dimuons are required to pass a \ctau selection that rejects dimuons with \ctau values above a given threshold. The \ctau selection threshold is optimized using simulations, keeping 90\% of prompt charmonia while rejecting more than 80\% of nonprompt charmonia. The \PsiP-to-\JPsi yields ratio is extracted from data by fitting the \mMuMu distribution of dimuons passing the \ctau selection and the ratio of prompt \PsiP over \JPsi meson yields is determined by subtracting the remaining nonprompt charmonia passing the \ctau selection.

The yields of prompt and nonprompt \JPsi mesons are found to be suppressed in all measurements. The nuclear modification factor (\raa) of \JPsi mesons is observed to depend on centrality, being more suppressed towards more central collisions, while no significant dependence on rapidity is seen. On the one hand, an indication of weaker suppression is observed for prompt \JPsi mesons, in the lowest transverse momentum (\pt) interval ($3 < \pt < 6.5$~\GeVc) and most central collisions (0-30\%), which may originate from \JPsi regeneration. On the other hand, the nonprompt \JPsi-meson suppression is seen to be more pronounced at high \pt, likely caused by jet quenching of bottom quarks. In the overlapping range, the measured \JPsi-meson \raa is compatible with previous measurements at $\sqrtsnn = \SI{2.76}{\TeV}$.

The results of the prompt charmonium and $\text{D}^{0}$-meson \raa are found to be compatible, in particular for $\pt > 15$~\GeVc, where the \raa is raising as \pt grows. This rise is understood in the context of jet quenching: more and more energetic partons lose a smaller fraction of their energy. This suggests that energy loss should play a role for \JPsi mesons, as for any other hadron. High-\pt prompt \JPsi mesons must partly arise from the fragmentation of gluons or color-octet quarkonium states, which are subject to energy loss effects induced by the QGP medium.

The measurement of the \doubleRatio double ratio in \RunPbPb at $\sqrtsnn = \SI{5.02}{\TeV}$ shows that the \PsiP mesons are more suppressed than \JPsi mesons, which is consistent with the sequential suppression of charmonia in the QGP. Comparisons with measurements at $\sqrtsnn = \SI{2.76}{\TeV}$ show a good agreement of the double ratio at high \pt in the mid-rapidity region. On the contrary, extending the \pt range down to 3~\GeVc in the forward rapidity region shows a stronger reduction of the double ratio at $\sqrtsnn = \SI{5,02}{\TeV}$ compared to $\sqrtsnn = \SI{2.76}{\TeV}$. A sequential regeneration of charmonia has been suggested to explain the double ratio results at both collision energies.